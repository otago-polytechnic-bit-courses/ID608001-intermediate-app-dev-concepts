% Author: Grayson Orr
% Course: ID608001: Intermediate Application Development Concepts

\documentclass{article}
\author{}

\usepackage{graphicx}
\usepackage{wrapfig}
\usepackage{enumerate}
\usepackage{hyperref}
\usepackage[margin = 1.75cm]{geometry}
\usepackage[table]{xcolor}
\usepackage{fancyhdr}
\hypersetup{
  colorlinks = true,
  urlcolor = blue 
} 
\setlength\parindent{0pt}
\pagestyle{fancy}
\fancyhf{}
\rhead{College of Engineering, Construction and Living Sciences\\Bachelor of Information Technology}
\lfoot{Project 1: Next.js Hacker News App \\Version 2, Semester Two, 2023}
\rfoot{\thepage}
 
\begin{document}

\begin{figure}
	\centering
	\includegraphics[width=50mm]{../../resources/img/logo.png}
\end{figure}

\title{College of Engineering, Construction and Living Sciences\\Bachelor of Information Technology\\ID608001: Intermediate Application Development Concepts\\Level 6, Credits 15\\\textbf{Project 1: Next.js Hacker News App}}
\date{}
\maketitle

\section*{Assessment Overview}
In this \textbf{individual} assessment, you will develop an application using \textbf{Next.js} and the \textbf{Hacker News API}. The main purpose of this assessment is to demonstrate your ability to develop an application using various taught concepts. In addition, marks will be allocated for code elegance, documentation and \textbf{Git} usage.

\section*{Learning Outcome}
At the successful completion of this course, learners will be able to:
\begin{enumerate}
	\item Apply design patterns and programming principles using software development best practices.
	\item Design and implement full-stack applications using industry relevant programming languages.
\end{enumerate}

\section*{Assessments}
\renewcommand{\arraystretch}{1.5}
\begin{tabular}{|c|c|c|c|}
	\hline
	\textbf{Assessment}                                 & \textbf{Weighting} & \textbf{Due Date}            & \textbf{Learning Outcomes} \\ \hline
	\small Practical: Skills-Based & \small 20\%        & \small 31-03-2023 (Fri at 10.00 AM)   & \small 1                   \\ \hline
	\small Project 1: Next.js Hacker News App             & \small 30\%        & \small 27-04-2023 (Thur at 4.59 PM)  & \small 1 and 2                   \\ \hline
	\small Project 2: Node.js and Express Pub Quiz App                       & \small 50\%        & \small 15-06-2023 (Thur at 4.59 PM)  & \small 1 and 2                   \\ \hline
\end{tabular}

\section*{Conditions of Assessment}
You will complete majority of this assessment during your learner-managed time. However, there will be time during class to discuss the requirements and your progress on this assessment. This assessment will need to be completed by \textbf{Thursday, 27 April 2023} at \textbf{4.59 PM}.

\section*{Pass Criteria}
This assessment is criterion-referenced (CRA) with a cumulative pass mark of \textbf{50\%} over all assessments in \textbf{ID608001: Intermediate Application Development Concepts}.

\section*{Authenticity}
All parts of your submitted assessment \textbf{must} be completely your work. Do your best to complete this assessment without using an \textbf{AI generative tool}. You need to demonstrate to the course lecturer that you can meet the learning outcome for this assessment. \\
 
 However, if you get stuck, you can use an \textbf{AI generative tool} to help you get unstuck, permitting you to acknowledge that you have used it. In the assessment's repository \textbf{README.md} file, please include what prompt(s) you provided to the \textbf{AI generative tool} and how you used the response(s) to help you with your work. It also applies to code snippets retrieved from \textbf{StackOverflow} and \textbf{GitHub}. \\
 
 Failure to do this may result in a mark of \textbf{zero} for this assessment.

\section*{Policy on Submissions, Extensions, Resubmissions and Resits}
The school's process concerning submissions, extensions, resubmissions and resits complies with \textbf{Otago Polytechnic} policies. Learners can view policies on the \textbf{Otago Polytechnic} website located at \href{https://www.op.ac.nz/about-us/governance-and-management/policies}{https://www.op.ac.nz/about-us/governance-and-management/policies}.

\section*{Submission}
You \textbf{must} submit all project files via \textbf{GitHub Classroom}. Here is the URL to the repository you will use for your submission – \href{https://classroom.github.com/a/T9vHopU9}{https://classroom.github.com/a/T9vHopU9}.  Create a \textbf{.gitignore} and add the ignored files in this resource - \href{https://raw.githubusercontent.com/github/gitignore/main/Node.gitignore}{https://raw.githubusercontent.com/github/gitignore/main/Node.gitignore}. The latest project files in the \textbf{master} or \textbf{main} branch will be used to mark against the \textbf{Functionality} criterion. Please test before you submit. Partial marks \textbf{will not} be given for incomplete functionality. Late submissions will incur a \textbf{10\% penalty per day}, rolling over at \textbf{5:00 PM}.

\section*{Extensions}
Familiarise yourself with the assessment due date. If you need an extension, contact the course lecturer before the due date. If you require more than a week's extension, a medical certificate or support letter from your manager may be needed.

\section*{Resubmissions}
Learners may be requested to resubmit an assessment following a rework of part/s of the original assessment. Resubmissions are to be completed within a negotiable short time frame and usually \textbf{must} be completed within the timing of the course to which the assessment relates. Resubmissions will be available to learners who have made a genuine attempt at the first assessment opportunity and achieved a \textbf{D grade (40-49\%)}. The maximum grade awarded for resubmission will be \textbf{C-}.

\section*{Resits}
Resits and reassessments \textbf{are not} applicable in \textbf{ID608001: Intermediate Application Development Concepts}.

\section*{Instructions}
You will need to submit an application and documentation that meet the following requirements:

\subsection*{Functionality - Learning Outcomes 1, 2, 3 (50\%)}
\begin{itemize}
	\item \textbf{Application:}
\begin{itemize}
	\item Display a drop down with the following options:
	\begin{itemize}
		\item Ask Stories
		\item Best Stories
		\item Job Stories
		\item New Stories
		\item Show Stories
		\item Top Stories
	\end{itemize}
	\item The endpoints for the stories are as follows:
	\begin{itemize}
		\item Ask Stories - \href{https://hacker-news.firebaseio.com/v0/askstories.json?print=pretty}{https://hacker-news.firebaseio.com/v0/askstories.json?print=pretty}
		\item Best Stories - \href{https://hacker-news.firebaseio.com/v0/beststories.json?print=pretty}{https://hacker-news.firebaseio.com/v0/beststories.json?print=pretty}
		\item Job Stories - \href{https://hacker-news.firebaseio.com/v0/jobstories.json?print=pretty}{https://hacker-news.firebaseio.com/v0/jobstories.json?print=pretty}
		\item New Stories - \href{https://hacker-news.firebaseio.com/v0/newstories.json?print=pretty}{https://hacker-news.firebaseio.com/v0/newstories.json?print=pretty}
		\item Show Stories - \href{https://hacker-news.firebaseio.com/v0/showstories.json?print=pretty}{https://hacker-news.firebaseio.com/v0/showstories.json?print=pretty}
		\item Top Stories - \href{https://hacker-news.firebaseio.com/v0/topstories.json?print=pretty}{https://hacker-news.firebaseio.com/v0/topstories.json?print=pretty}
	\end{itemize}
	\item When an option is selected, display the title of the first 40 stories.
	\begin{itemize}
		\item Style each story to look like a card using \textbf{Tailwind CSS}.
		\item Display five stories per row.
	\end{itemize}
	\item When a story is clicked, the user will be navigate to a page that displays the following information:
	\begin{itemize}
		\item By
		\item Kids. \textbf{Note:} This is an array of ids. If the array is empty, display \textbf{N/A}. Display the first five ids as URLs in this format: \href{https://hacker-news.firebaseio.com/v0/item/$<$Id$>$.json?print=pretty}{https://hacker-news.firebaseio.com/v0/item/$<$Id$>$.json?print=pretty}. When clicked, these URLs will open in a new tab. 
		\item Score
		\item Time. \textbf{Note:} Convert the time to a readable format.
		\item Title
		\item Type 
		\item URL. \textbf{Note:} When clicked, this URL will open in a new tab.
	\end{itemize}
	This information is fetched from the following endpoint:\\
	\href{https://hacker-news.firebaseio.com/v0/item/$<$Id$>$.json?print=pretty}{https://hacker-news.firebaseio.com/v0/item/$<$Id$>$.json?print=pretty}.
	\item On a page, display the top 20 leaders. You can get this information from the following URL:\\
	\href{https://news.ycombinator.com/leaders}{https://news.ycombinator.com/leaders}. \textbf{Note:} You need to manually retrieve the information from the URL. 
	\item When a leader is clicked, the user will be navigate to a page that displays the following information:
	\begin{itemize}
		\item About
		\item Created. \textbf{Note:} Convert the time to a readable format.
		\item Id
		\item Karma
		\item Submitted. \textbf{Note:} This is an array of ids. Display the first 10 ids as URLs in this format:\\
		 \href{https://hacker-news.firebaseio.com/v0/item/$<$Id$>$.json?print=pretty}{https://hacker-news.firebaseio.com/v0/item/$<$Id$>$.json?print=pretty}.
	\end{itemize}
	This information is fetched from the following endpoint:\\
	\href{https://hacker-news.firebaseio.com/v0/user/$<$Id$>$.json?print=pretty}{https://hacker-news.firebaseio.com/v0/user/$<$Id$>$.json?print=pretty}.
	\item Deployed your application to \textbf{Vercel}. 
\end{itemize}

\item \textbf{Testing:}
\begin{itemize}
	\item \textbf{Component tests} are written using \textbf{React Testing Library} and \textbf{Jest}.
	\item 15 \textbf{component tests} verifying the app functionality.
\end{itemize}

\item \textbf{Scripts:}
\begin{itemize}
	\item Run your \textbf{component tests} using \textbf{React Testing Library} and \textbf{Jest}.
	\item Lint and fix your code using \textbf{ESLint}.
	\item Format your code using \textbf{Prettier}.
\end{itemize}
\end{itemize}

\subsection*{Code Elegance - Learning Outcome 1 (40\%)}
\begin{itemize}
	\item A \textbf{Node.js} \textbf{.gitignore} file is used.
	\item Appropriate naming of files, variables, functions and components.
	\item Idiomatic use of control flow, data structures and in-built functions.
	\item Efficient algorithmic approach.
	\item Sufficient modularity.
	\item Each \textbf{component} and \textbf{page} file \textbf{must} have a \textbf{JSDoc} header comment located immediately before the \textbf{import} statements.
	\item In-line comments where required. It should be for code that needs further explanation.
	\item Code is linted and formatted using \textbf{ESLint} and \textbf{Prettier}.
	\item \textbf{React Testing Library}, \textbf{Jest}, \textbf{ESLint}, \textbf{Prettier} and \textbf{Commitizen} are installed as \textbf{development dependencies}.	
	\item No dead or unused code.
\end{itemize}

\subsection*{Documentation and Git/GitHub Usage - Learning Outcomes 2, 3 (10\%)}
\begin{itemize}
	\item \textbf{GitHub} project board to help you organise and prioritise your work. 
	\item Provide the following in your repository \textbf{README.md} file:
	\begin{itemize}
		\item A URL to your application on \textbf{Vercel}.
		\item How do you setup the environment, i.e., after the repository is cloned, what do you need to do before you run the application?
		\item How do you run the application locally?
		\item How do you run your \textbf{component tests}?
		\item How do you lint and fix your code?
		\item How do you format your code?
	\end{itemize}
	\item Use of \textbf{Markdown}, i.e., headings, bold text, code blocks, etc.
	\item Correct spelling and grammar. 
	\item Your \textbf{Git commit messages} should:
	\begin{itemize}
		\item Reflect the context of each functional requirement change.
		\item Be formatted using an appropriate naming convention style using \textbf{Commitizen}.
	\end{itemize}	
\end{itemize}

\subsection*{Additional Information}
\begin{itemize}
	\item \textbf{Do not} rewrite your \textbf{Git} history. It is important that the course lecturer can see how you worked on your assessment over time.
\end{itemize}

\end{document}
