% Author: Grayson Orr
% Course: ID608001: Intermediate Application Development Concepts

\documentclass{article}
\author{}

\usepackage{graphicx}
\usepackage{wrapfig}
\usepackage{enumerate}
\usepackage{hyperref}
\usepackage[margin = 2.25cm]{geometry}
\usepackage[table]{xcolor}
\usepackage{fancyhdr}
\hypersetup{
  colorlinks = true,
  urlcolor = blue
}
\setlength\parindent{0pt}
\pagestyle{fancy}
\fancyhf{}
\rhead{College of Engineering, Construction and Living Sciences\\Bachelor of Information Technology}
\lfoot{Assessment 1: Node.js RESTful API - Open Trivia DB\\Version 1, Semester Two, 2022}
\rfoot{\thepage}
 
\begin{document}

\begin{figure}
	\centering
	\includegraphics[width=50mm]{../../resources/img/logo.png}
\end{figure}

\title{College of Engineering, Construction and Living Sciences\\Bachelor of Information Technology\\ID608001: Intermediate Application Development Concepts\\Level 6, Credits 15\\\textbf{Assessment 1: Node.js RESTful API - Open Trivia DB}}
\date{}
\maketitle

\section*{Assessment Overview}

\section*{Learning Outcome}
At the successful completion of this course, learners will be able to:
\begin{enumerate}
	\item Apply design patterns \& programming principles using software development best practices.
	\item Design \& implement full-stack applications using industry relevant programming languages.
\end{enumerate}

\section*{Assessments}
\renewcommand{\arraystretch}{1.5}
\begin{tabular}{|c|c|c|c|}
	\hline
	\textbf{Assessment}                                 & \textbf{Weighting} & \textbf{Due Date}            & \textbf{Learning Outcomes} \\ \hline
	\small Practical: Skills-Based & \small 20\%        & \small 27-10-2022 (Wed at 7.59 AM)   & \small 1                   \\ \hline
	\small Assessment 1: Node.js RESTful API - Open Trivia DB              & \small 40\%        & \small 17-10-2022 (Mon at 7.59 AM)  & \small 1 \& 2                   \\ \hline
	\small Assessment 2: React - Open Trivia DB                        & \small 40\%        & \small 09-11-2022 (Wed at 2.59 PM)  & \small 1 \& 2                   \\ \hline
\end{tabular}

\section*{Conditions of Assessment}
You will complete this assessment during your learner-managed time. However, there will be time during class to discuss the requirements \& your progress on this assessment. This assessment will need to be completed by \textbf{Monday, 17 October 2022} at \textbf{7.59 AM}.

\section*{Pass Criteria}
This assessment is criterion-referenced (CRA) with a cumulative pass mark of \textbf{50\%} over all assessments in \textbf{ID608001: Intermediate Application Development Concepts}.

\section*{Authenticity}
All parts of your submitted assessment \textbf{must} be completely your work. If you use code snippets from \textbf{GitHub}, \textbf{StackOverflow} or other online resources, you \textbf{must} reference it appropriately using \textbf{APA 7th edition}. Provide your references in the \textbf{README.md} file in your repository. Failure to do this will result in a mark of \textbf{zero} for this assessment.

\section*{Policy on Submissions, Extensions, Resubmissions \& Resits}
The school's process concerning submissions, extensions, resubmissions \& resits complies with \textbf{Otago Polytechnic} policies. Learners can view policies on the \textbf{Otago Polytechnic} website located at \href{https://www.op.ac.nz/about-us/governance-and-management/policies}{https://www.op.ac.nz/about-us/governance-and-management/policies}.

\section*{Submission}
You \textbf{must} submit all project files via \textbf{GitHub Classroom}. Here is the URL to the repository you will use for your submission – \href{https://classroom.github.com/a/i4G4NwNS}{https://classroom.github.com/a/i4G4NwNS}.  Create a \textbf{.gitignore} \& add the ignored files in this resource - \href{https://raw.githubusercontent.com/github/gitignore/main/Node.gitignore}{https://raw.githubusercontent.com/github/gitignore/main/Node.gitignore}. The latest project files in the \textbf{master} or \textbf{main} branch will be used to mark against the \textbf{Functionality} criterion. Please test before you submit. Partial marks \textbf{will not} be given for incomplete functionality. Late submissions will incur a \textbf{10\% penalty per day}, rolling over at \textbf{8:00 AM}.

\section*{Extensions}
Familiarise yourself with the assessment due date. If you need an extension, contact the course lecturer before the due date. If you require more than a week's extension, a medical certificate or support letter from your manager may be needed.

\section*{Resubmissions}
Learners may be requested to resubmit an assessment following a rework of part/s of the original assessment. Resubmissions are to be completed within a negotiable short time frame \& usually \textbf{must} be completed within the timing of the course to which the assessment relates. Resubmissions will be available to learners who have made a genuine attempt at the first assessment opportunity \& achieved a \textbf{D grade (40-49\%)}. The maximum grade awarded for resubmission will be \textbf{C-}.

\section*{Resits}
Resits \& reassessments \textbf{are not} applicable in \textbf{ID721001: Mobile Application Development}.

\section*{Instructions}
You will need to submit an application \& documentation that meet the following requirements:

\subsection*{Functionality - Learning Outcomes 1, 2, 3 (50\%)}
\begin{itemize}
	\item User:
	\begin{itemize}
		\item You will have three types of users - super admin, admin and basic user.
		\item Each user will have the following information: first name, last name, username, email address, profile picture, password, confirm password and role. The users' profile picture will be fetched from the following API - \href{https://avatars.dicebear.com/docs/http-api}{https://avatars.dicebear.com/docs/http-api} using Axios. 
		\item Each user can login, logout, get their information and update their information. A super admin user can get all users' information, update all admin and basic users' information and delete all admin and basic users. An admin user can get all admin and basic users' information and update all basic users' information. A basic user can register.
		\item When performing a POST request for registering a basic user, the following error checking must be implemented:
		\begin{itemize}
			\item First name has a minimum length of two characters, a maximum length of 50 characters and alpha characters only.
			\item Last name has the same error checking as first name above.
			\item Username is unique, has a minimum length of five characters, maximum length of ten characters and alphanumeric characters only, i.e., johndoe123.
			\item Email address is unique, contains the username above, an @ special character and a second-level domain, i.e., johndoe123@email.com.
			\item Password has a minimum length of eight characters, maximum length of 16 characters and contains one numeric character and one special charcater.
			\item Confirm password is the same as the password above.
		\end{itemize}
		For each error check, a status code and response message is returned, i.e., "First name must have a minimum length of two characters".
		\item When performing a POST request for logging in a user, return a status code, a response message, i.e., "<User's username> has successfully logged in" and the user's JWT.
		\item When performing a GET request for logging out a user, return a status code, a response message, i.e., "<User's username> has successfully logged out" and set the user's JWT to expired.
		\item When performing a PUT and DELETE request, return a status code and a response message, i.e., "<User's username>'s information has successfully updated" or "<User's username> has successfully deleted".
		\item Two super admin users are seeded via you. Only you can seed the two super admin users. The admin users' data will be fetched from a local file and inserted into the User table using Prisma. 
		\item Five admin users are seeded via a super admin user. Only a super admin user can seed the five admin users. The admin users' data will be fetched from a private GitHub Gist using Axios and inserted into the User table using Prisma.  
		\item Five basic users are seeded via a super admin or an admin user. Only a super admin or an admin user can seed the five basic users. The basic users' data will be fetched from a private GitHub Gist using Axios and inserted into the User table using Prisma. 
	\end{itemize} 
	\item Quiz:
	\begin{itemize}
		\item Each quiz will have the following information: name, start date, end date, category, difficulty, type, number of questions, list of questions, list of correct answers, list of incorrect answers, list of scores, average score, list of ratings and average rating. The category, list of questions, list of correct answers and list of incorrect answers will be fetched from the following API - \href{https://opentdb.com/api\_config.php}{https://opentdb.com/api\_config.php}. The difficulties will be easy, medium and hard. The types will be multiple choice or true/false.
		\item Each user can get all quizzes, get all past quizzes, get all present quizzes and get all future quizzes. A super admin and an admin user can create a quiz. A super admin user can delete a quiz. A basic user can participate in a quiz and rate a quiz.
		\item When performing a POST request for creating a quiz, the following error checking must be implemented:
		\begin{itemize}
			\item Name has a minimum length of five characters, a maximum length of 30 characters and alpha characters only.
			\item Start date has to greater than today's date.
			\item End date has to greater than the start date and no longer than five days. 
			\item Number of questions has to be ten.
		\end{itemize}
		For each error check, a status code and response message is returned, i.e., "Name must have a minimum length of five characters".
		\item When performing a POST request for a basic user who has participated in a quiz, return a status code, a response message, i.e., "<User's username> has successfully participated in <Quiz's name>", user's score and quiz's average score.
	\end{itemize}
\end{itemize}

\subsection*{Code Elegance - Learning Outcome 1 (30\%)}
\begin{itemize}
	\item Environment variables' key is stored in the example.env file. 
	\item Database configured for development and production environments.
	\item Variables, functions and resource groups are named appropriately.
	\item Idiomatic use of control flow, data structures and in-built functions.
	\item A file header comment for each controller and route file explaining its purpose using JSDoc.
	\item Code is formatted.
	\item No unused code.
\end{itemize}

\subsection*{Documentation \& Git/GitHub Usage - Learning Outcomes 2, 3 (20\%)}
% \begin{itemize}
	
% \end{itemize}

\subsection*{Additional Information}
\begin{itemize}
	\item \textbf{Do not} rewrite your \textbf{Git} history. It is important that the course lecturer can see how you worked on your assessment over time.
\end{itemize}

\end{document}