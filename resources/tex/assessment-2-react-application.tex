% Author: Grayson Orr
% Course: ID608001: Intermediate Application Development Concepts

\documentclass{article}
\author{}

\usepackage{graphicx}
\usepackage{wrapfig}
\usepackage{enumerate}
\usepackage{hyperref}
\usepackage[margin = 1.75cm]{geometry}
\usepackage[table]{xcolor}
\usepackage{fancyhdr}
\hypersetup{
  colorlinks = true,
  urlcolor = blue 
}
\setlength\parindent{0pt}
\pagestyle{fancy}
\fancyhf{}
\rhead{College of Engineering, Construction \& Living Sciences\\Bachelor of Information Technology}
\lfoot{Assessment 2: React Application - The Movie DB\\Version 1, Semester Two, 2022}
\rfoot{\thepage}
 
\begin{document}

\begin{figure}
	\centering
	\includegraphics[width=50mm]{../../resources/img/logo.png}
\end{figure}

\title{College of Engineering, Construction \& Living Sciences\\Bachelor of Information Technology\\ID608001: Intermediate Application Development Concepts\\Level 6, Credits 15\\\textbf{Assessment 2: React Application - The Movie DB}}
\date{}
\maketitle

\section*{Assessment Overview}
In this \textbf{individual} assessment, you will replicate a given application using \textbf{React}. The main purpose of this assessment is to demonstrate your ability to replicate an existing application using various taught frontend concepts. However, you will be required to independently research \& implement more complex concepts. In addition, marks will be allocated for code elegance, documentation \& \textbf{Git} usage.

\section*{Learning Outcome}
At the successful completion of this course, learners will be able to:
\begin{enumerate}
	\item Apply design patterns \& programming principles using software development best practices.
	\item Design \& implement full-stack applications using industry relevant programming languages.
\end{enumerate}

\section*{Assessments}
\renewcommand{\arraystretch}{1.5}
\begin{tabular}{|c|c|c|c|}
	\hline
	\textbf{Assessment}                                 & \textbf{Weighting} & \textbf{Due Date}            & \textbf{Learning Outcomes} \\ \hline
	\small Practical: Skills-Based & \small 20\%        & \small 27-10-2022 (Thur at 10.00 AM)   & \small 1                   \\ \hline
	\small Assessment 1: Node.js RESTful API - Open Trivia DB              & \small 45\%        & \small 17-10-2022 (Mon at 7.59 AM)  & \small 1 \& 2                   \\ \hline
	\small Assessment 2: React Application - The Movie DB                       & \small 35\%        & \small 09-11-2022 (Wed at 2.59 PM)  & \small 1 \& 2                   \\ \hline
\end{tabular}

\section*{Conditions of Assessment}
You will complete this assessment during your learner-managed time. However, there will be time during class to discuss the requirements \& your progress on this assessment. This assessment will need to be completed by \textbf{Wednesday, 09 November 2022} at \textbf{2.59 PM}.

\section*{Pass Criteria}
This assessment is criterion-referenced (CRA) with a cumulative pass mark of \textbf{50\%} over all assessments in \textbf{ID608001: Intermediate Application Development Concepts}.

\section*{Authenticity}
All parts of your submitted assessment \textbf{must} be completely your work. If you use code snippets from \textbf{GitHub}, \textbf{StackOverflow} or other online resources, you \textbf{must} reference it appropriately using \textbf{APA 7th edition}. Provide your references in the \textbf{README.md} file in your repository. Failure to do this will result in a mark of \textbf{zero} for this assessment.

\section*{Policy on Submissions, Extensions, Resubmissions \& Resits}
The school's process concerning submissions, extensions, resubmissions \& resits complies with \textbf{Otago Polytechnic} policies. Learners can view policies on the \textbf{Otago Polytechnic} website located at \href{https://www.op.ac.nz/about-us/governance-and-management/policies}{https://www.op.ac.nz/about-us/governance-and-management/policies}.

\section*{Submission}
You \textbf{must} submit all project files via \textbf{GitHub Classroom}. Here is the URL to the repository you will use for your submission – \href{https://classroom.github.com/a/Ys5M8rKj}{https://classroom.github.com/a/Ys5M8rKj}.  Create a \textbf{.gitignore} \& add the ignored files in this resource - \href{https://raw.githubusercontent.com/github/gitignore/main/Node.gitignore}{https://raw.githubusercontent.com/github/gitignore/main/Node.gitignore}. The latest project files in the \textbf{master} or \textbf{main} branch will be used to mark against the \textbf{Functionality} criterion. Please test before you submit. Partial marks \textbf{will not} be given for incomplete functionality. Late submissions will incur a \textbf{10\% penalty per day}, rolling over at \textbf{3.00 PM}.

\section*{Extensions}
Familiarise yourself with the assessment due date. If you need an extension, contact the course lecturer before the due date. If you require more than a week's extension, a medical certificate or support letter from your manager may be needed.
 
\section*{Resubmissions}
Learners may be requested to resubmit an assessment following a rework of part/s of the original assessment. Resubmissions are to be completed within a negotiable short time frame \& usually \textbf{must} be completed within the timing of the course to which the assessment relates. Resubmissions will be available to learners who have made a genuine attempt at the first assessment opportunity \& achieved a \textbf{D grade (40-49\%)}. The maximum grade awarded for resubmission will be \textbf{C-}.

\section*{Resits}
Resits \& reassessments \textbf{are not} applicable in \textbf{ID608001: Intermediate Application Development Concepts}.

\section*{Instructions}
You will need to submit an application \& documentation that meet the following requirements:

\subsection*{Functionality - Learning Outcomes 1, 2, 3 (50\%)}
\begin{itemize}
	\item \textbf{Header:}
	\begin{itemize}
		\item The \textbf{Header} component will render six icon components using the \textbf{@material-ui/core@4.11.2} \& \textbf{@material-ui/icons@4.11.2} dependencies. The components are:
		\begin{itemize}
			\item Home
			\item FlashOn
			\item LiveTv
			\item PersonOutline
			\item Search
			\item VideoLibrary
		\end{itemize} 
		\item When the cursor hovers over an icon, it will display the following text:
		\begin{itemize}
			\item Home icon - Home
			\item FlashOn icon - Trending
			\item LiveTv - Verified
			\item PersonOutline - Account
			\item Search - Search
			\item VideoLibrary - Collections
		\end{itemize}
		\item \textbf{Resources:} 
		\begin{itemize}
			\item \href{https://www.npmjs.com/package/@material-ui/core}{https://www.npmjs.com/package/@material-ui/core}
			\item \href{https://www.npmjs.com/package/@material-ui/icons}{https://www.npmjs.com/package/@material-ui/icons}
		\end{itemize}
	\end{itemize}
	\item \textbf{Navigation Bar:}
	\begin{itemize}
		\item The \textbf{Navbar} component will render eight items fetched from the given file - \textbf{endpoint.js}. The \textbf{endpoint.js} file is located in the given directory called \textbf{utils}. In the \textbf{endpoint.js}, refer to the \textbf{type} key.
		\item The eight items need to be centered.
		\item When the cursor hovers over an item, it will change it's colour and font size.
	\end{itemize}
	\item \textbf{Movie Card:}
	\begin{itemize}
		\item You have been given the incomplete \textbf{MovieCard} component file. Using the given comments in the \textbf{MovieCard} component file, complete the following functionality:
		\begin{itemize}
			\item Under the \textbf{img} element, render the movie's overview using the \textbf{react-text-truncate@0.16.0} dependency. 
			\item Under the \textbf{h2} element, render the movie's media type, release date or first air date \& vote count in a \textbf{span} element. 
		\end{itemize}
		\item For styling, the \textbf{MovieCard} component uses the \textbf{react-jss@10.5.0} dependency.
		\item \textbf{Resources:}
		\begin{itemize}
			\item \href{https://www.npmjs.com/package/react-text-truncate}{https://www.npmjs.com/package/react-text-truncate}
			\item \href{https://www.npmjs.com/package/react-jss}{https://www.npmjs.com/package/react-jss}
		\end{itemize}
	\end{itemize}
	\item \textbf{Dashboard:}
	\begin{itemize}
		\item The \textbf{Dashboard} component will fetch a list of movies using the \textbf{axios} dependency from the given file - \textbf{endpoint.js}. In the \textbf{utils} directory, you have been given a file called \textbf{axios.js}. This file creates a new instance of \textbf{axios} with the base URL - \href{https://api.themoviedb.org/3}{https://api.themoviedb.org/3}. In the \textbf{endpoint.js}, refer to the \textbf{url} key.
		\item For each movie in the list of movies, render the \textbf{MovieCard} component.
	\end{itemize}
	\item \textbf{App:}
	\begin{itemize}
		\item The \textbf{App} component will render the \textbf{Header} component, \textbf{Navbar} component \& \textbf{Dashboard} component.
	\end{itemize}
\end{itemize}

\subsection*{Code Elegance - Learning Outcome 1 (35\%)}
\begin{itemize}
	\item Environment variables' key is stored in the \textbf{env.example} file. 
	\item Variables, functions \& components are named appropriately.
	\item Idiomatic use of control flow, data structures \& in-built functions.
	\item Sufficient modularity, i.e., UI split into independent reusable pieces.
	\item File header comment for each component file explaining its purpose using \textbf{JSDoc}.
	\item Code is linted \& formatted using \textbf{ESLint} \& \textbf{Prettier}.
	\item \textbf{ESLint}, \textbf{Prettier} \& \textbf{Commitizen} are installed as \textbf{development dependencies}.	
\end{itemize}

\subsection*{Documentation \& Git/GitHub Usage - Learning Outcomes 2, 3 (15\%)}
\begin{itemize}
	\item \textbf{GitHub} project board to help you organise \& prioritise your work. 
	\item Provide the following in your repository \textbf{README.md} file:
	\begin{itemize}
		\item How do you setup the development environment, i.e., after the repository is cloned, what do you need to do before you run the \textbf{React} application?
		\item How do you lint \& fix your code?
		\item How do you format your code?
	\end{itemize}
	\item Use of \textbf{Markdown}, i.e., headings, bold text, code blocks, etc.
	\item Correct spelling \& grammar. 
	\item Your \textbf{Git commit messages} should:
	\begin{itemize}
		\item Reflect the context of each functional requirement change.
		\item Be formatted using an appropriate naming convention style using \textbf{Commitizen}.
	\end{itemize}	
\end{itemize}

\subsection*{Additional Information}
\begin{itemize}
	\item In this repository, you have been given a directory called \textbf{expected-outputs}. In this directory, you are given expected output screenshots for this assessment
	\item \textbf{Do not} rewrite your \textbf{Git} history. It is important that the course lecturer can see how you worked on your assessment over time.
\end{itemize}

\end{document}