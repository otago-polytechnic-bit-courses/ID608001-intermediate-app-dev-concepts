% Author: Grayson Orr
% Course: ID608001: Intermediate Application Development Concepts

\documentclass{article}
\author{}

\usepackage{graphicx}
\usepackage{wrapfig}
\usepackage{enumerate}
\usepackage{hyperref}
\usepackage[margin = 1.75cm]{geometry}
\usepackage[table]{xcolor}
\usepackage{fancyhdr}
\hypersetup{
  colorlinks = true,
  urlcolor = blue 
}
\setlength\parindent{0pt}
\pagestyle{fancy}
\fancyhf{}
\rhead{College of Engineering, Construction \& Living Sciences\\Bachelor of Information Technology}
\lfoot{Assessment 2: Next.js Application\\Version 1, Semester Two, 2022}
\rfoot{\thepage}
 
\begin{document}

\begin{figure}
	\centering
	\includegraphics[width=50mm]{../../resources/img/logo.png}
\end{figure}

\title{College of Engineering, Construction \& Living Sciences\\Bachelor of Information Technology\\ID608001: Intermediate Application Development Concepts\\Level 6, Credits 15\\\textbf{Assessment 2: Next.js Application}}
\date{}
\maketitle

\section*{Assessment Overview}
In this \textbf{individual} assessment, you will replicate the given application on \textbf{Heroku} using \textbf{Next.js}. The main purpose of this assessment is to demonstrate your ability to develop an application using various taught frontend concepts. However, you will be required to independently research \& implement more complex concepts. In addition, marks will be allocated for code elegance, documentation \& \textbf{Git} usage.

\section*{Learning Outcome}
At the successful completion of this course, learners will be able to:
\begin{enumerate}
	\item Apply design patterns \& programming principles using software development best practices.
	\item Design \& implement full-stack applications using industry relevant programming languages.
\end{enumerate}

\section*{Assessments}
\renewcommand{\arraystretch}{1.5}
\begin{tabular}{|c|c|c|c|}
	\hline
	\textbf{Assessment}                                 & \textbf{Weighting} & \textbf{Due Date}            & \textbf{Learning Outcomes} \\ \hline
	\small Practical: Skills-Based & \small 20\%        & \small 27-10-2022 (Thur at 10.00 AM)   & \small 1                   \\ \hline
	\small Assessment 2: Next.js Application              & \small 45\%        & \small 17-10-2022 (Mon at 7.59 AM)  & \small 1 \& 2                   \\ \hline
	\small Assessment 2: Next.js Application                       & \small 35\%        & \small 09-11-2022 (Wed at 2.59 PM)  & \small 1 \& 2                   \\ \hline
\end{tabular}

\section*{Conditions of Assessment}
You will complete this assessment during your learner-managed time. However, there will be time during class to discuss the requirements \& your progress on this assessment. This assessment will need to be completed by \textbf{Wednesday, 09 November 2022} at \textbf{2.59 PM}.

\section*{Pass Criteria}
This assessment is criterion-referenced (CRA) with a cumulative pass mark of \textbf{50\%} over all assessments in \textbf{ID608001: Intermediate Application Development Concepts}.

\section*{Authenticity}
All parts of your submitted assessment \textbf{must} be completely your work. If you use code snippets from \textbf{GitHub}, \textbf{StackOverflow} or other online resources, you \textbf{must} reference it appropriately using \textbf{APA 7th edition}. Provide your references in the \textbf{README.md} file in your repository. Failure to do this will result in a mark of \textbf{zero} for this assessment.

\section*{Policy on Submissions, Extensions, Resubmissions \& Resits}
The school's process concerning submissions, extensions, resubmissions \& resits complies with \textbf{Otago Polytechnic} policies. Learners can view policies on the \textbf{Otago Polytechnic} website located at \href{https://www.op.ac.nz/about-us/governance-and-management/policies}{https://www.op.ac.nz/about-us/governance-and-management/policies}.

\section*{Submission}
You \textbf{must} submit all project files via \textbf{GitHub Classroom}. Here is the URL to the repository you will use for your submission – \href{https://classroom.github.com/a/Ys5M8rKj}{https://classroom.github.com/a/Ys5M8rKj}.  Create a \textbf{.gitignore} \& add the ignored files in this resource - \href{https://raw.githubusercontent.com/github/gitignore/main/Node.gitignore}{https://raw.githubusercontent.com/github/gitignore/main/Node.gitignore}. The latest project files in the \textbf{master} or \textbf{main} branch will be used to mark against the \textbf{Functionality} criterion. Please test before you submit. Partial marks \textbf{will not} be given for incomplete functionality. Late submissions will incur a \textbf{10\% penalty per day}, rolling over at \textbf{3.00 PM}.

\section*{Extensions}
Familiarise yourself with the assessment due date. If you need an extension, contact the course lecturer before the due date. If you require more than a week's extension, a medical certificate or support letter from your manager may be needed.

\section*{Resubmissions}
Learners may be requested to resubmit an assessment following a rework of part/s of the original assessment. Resubmissions are to be completed within a negotiable short time frame \& usually \textbf{must} be completed within the timing of the course to which the assessment relates. Resubmissions will be available to learners who have made a genuine attempt at the first assessment opportunity \& achieved a \textbf{D grade (40-49\%)}. The maximum grade awarded for resubmission will be \textbf{C-}.

\section*{Resits}
Resits \& reassessments \textbf{are not} applicable in \textbf{ID608001: Intermediate Application Development Concepts}.

\section*{Instructions}
You will need to submit an application \& documentation that meet the following requirements:

\subsection*{Functionality - Learning Outcomes 1, 2, 3 (50\%)}
\begin{itemize}
	
\end{itemize}

\subsection*{Code Elegance - Learning Outcome 1 (35\%)}
\begin{itemize}
	\item Environment variables' key is stored in the \textbf{env.example} file. 
	\item Variables, functions \& components are named appropriately.
	\item Idiomatic use of control flow, data structures \& in-built functions.
	\item Sufficient modularity, i.e., UI split into independent reusable pieces.
	\item File header comment for each component explaining its purpose using \textbf{JSDoc}.
	\item Code is linted \& formatted using \textbf{ESLint} \& \textbf{Prettier}.
	\item \textbf{ESLint}, \textbf{Prettier} \& \textbf{Commitizen} are installed as \textbf{development dependencies}.	
\end{itemize}

\subsection*{Documentation \& Git/GitHub Usage - Learning Outcomes 2, 3 (15\%)}
\begin{itemize}
	\item \textbf{GitHub} project board to help you organise \& prioritise your work. 
	\item Provide the following in your repository \textbf{README.md} file:
	\begin{itemize}
		\item A \textbf{Entity-Relationship} diagram of your \textbf{Prisma} schema.
		\item URL to the \textbf{RESTful API} on \textbf{Heroku}.
		\item How do you setup the development environment, i.e., after the repository is cloned, what do you need to do before you run the \textbf{RESTful API}?
		\item How do you deploy the \textbf{RESTful API} to \textbf{Heroku}
		\item How do you open \textbf{Prisma Studio}?
		\item How do you create a migration?
		\item How do you lint \& fix your code?
		\item How do you format your code?
		\item How do you run your \textbf{API/integration} tests?
		\item How do you run your code coverage \& output the results to \textbf{HTML}?
	\end{itemize}
	\item Use of \textbf{Markdown}, i.e., headings, bold text, code blocks, etc.
	\item Correct spelling \& grammar. 
	\item Your \textbf{Git commit messages} should:
	\begin{itemize}
		\item Reflect the context of each functional requirement change.
		\item Be formatted using an appropriate naming convention style using \textbf{Commitizen}.
	\end{itemize}	
\end{itemize}

\subsection*{Additional Information}
\begin{itemize}
	\item \textbf{Do not} rewrite your \textbf{Git} history. It is important that the course lecturer can see how you worked on your assessment over time.
\end{itemize}

\end{document}
