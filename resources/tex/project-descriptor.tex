% Author: Grayson Orr
% Course: ID608001: Intermediate Application Development Concepts

\documentclass{article}
\author{}

\usepackage{graphicx}
\usepackage{wrapfig}
\usepackage{enumerate}
\usepackage{hyperref}
\usepackage[margin = 1.75cm]{geometry}
\usepackage[table]{xcolor}
\usepackage{fancyhdr}
\hypersetup{
  colorlinks = true,
  urlcolor = blue 
} 
\setlength\parindent{0pt}
\pagestyle{fancy}
\fancyhf{}
\rhead{College of Engineering, Construction and Living Sciences\\Bachelor of Information Technology}
\lfoot{Project \\Version 2, Semester One, 2024}
\rfoot{\thepage}
 
\begin{document}

\begin{figure}
	\centering
	\includegraphics[width=50mm]{../../resources/img/logo.png}
\end{figure}

\title{College of Engineering, Construction and Living Sciences\\Bachelor of Information Technology\\ID608001: Intermediate Application Development Concepts\\Level 6, Credits 15\\\textbf{Project}}
\date{}
\maketitle

\section*{Assessment Overview}
In this \textbf{individual} assessment, you will develop a \textbf{RESTful API} using \textbf{Express}, \textbf{Node.js} and the \textbf{OpenTDB API}, and deploy it as a \textbf{web service} on \textbf{Render}. Your data will be stored in a \textbf{PostgreSQL} database on \textbf{Render}. Also, you will develop \textbf{frontend applications} using \textbf{React}. These applications will consume the \textbf{RESTful API} mentioned above. The main purpose of this assessment is not just to build \textbf{full-stack applications}, but rather to demonstrate an ability to decouple your \textbf{RESTful API} from your \textbf{frontend applications}. In addition, marks will be allocated for code quality and best practices, documentation and \textbf{Git} usage.

\section*{Learning Outcome}
At the successful completion of this course, learners will be able to:
\begin{enumerate}
	\item Apply design patterns and programming principles using software development best practices.
	\item Design and implement full-stack applications using industry relevant programming languages.
\end{enumerate}

\section*{Assessments}
\renewcommand{\arraystretch}{1.5}
\begin{tabular}{|c|c|c|c|}
	\hline
	\textbf{Assessment}                                 & \textbf{Weighting} & \textbf{Due Date}            & \textbf{Learning Outcome} \\ \hline
	\small Practical & \small 20\%        & \small 29-05-2024 (Wednesday at 4.59 PM)   & \small 1                   \\ \hline
	\small Project                 & \small 80\%        & \small 21-06-2024 (Friday at 4.59 PM) \small  & \small 1 and 2                   \\ \hline
\end{tabular}

\section*{Conditions of Assessment}
You will complete this assessment during your learner-managed time. However, there will be time during class to discuss the requirements and your progress on this assessment. This assessment will need to be completed by \textbf{Friday, 21 June 2024} at \textbf{4.59 PM}. 

\section*{Pass Criteria}
This assessment is criterion-referenced (CRA) with a cumulative pass mark of \textbf{50\%} over all assessments in \textbf{ID608001: Intermediate Application Development Concepts}.

\section*{Authenticity}
All parts of your submitted assessment \textbf{must} be completely your work. Do your best to complete this assessment without using an \textbf{AI generative tool}. You need to demonstrate to the course lecturer that you can meet the learning outcome(s) for this assessment. \\
 
 However, if you get stuck, you can use an \textbf{AI generative tool} to help you get unstuck, permitting you to acknowledge that you have used it. In the assessment's repository \textbf{README.md} file, please include what prompt(s) you provided to the \textbf{AI generative tool} and how you used the response(s) to help you with your work. It also applies to code snippets retrieved from \textbf{StackOverflow} and \textbf{GitHub}. \\
 
 Failure to do this may result in a mark of \textbf{zero} for this assessment.

\section*{Policy on Submissions, Extensions, Resubmissions and Resits}
The school's process concerning submissions, extensions, resubmissions and resits complies with \textbf{Otago Polytechnic | Te Pūkenga} policies. Learners can view policies on the \textbf{Otago Polytechnic | Te Pūkenga} website located at \href{https://www.op.ac.nz/about-us/governance-and-management/policies}{https://www.op.ac.nz/about-us/governance-and-management/policies}.

\section*{Submission}
You \textbf{must} submit all application files via \textbf{GitHub Classroom}. Here is the URL to the repository you will use for your submission – \href{https://classroom.github.com/a/wlzE5yYo}{https://classroom.github.com/a/wlzE5yYo}. If you do not have not one, create a \textbf{.gitignore} and add the ignored files in this resource - \href{https://raw.githubusercontent.com/github/gitignore/main/Node.gitignore}{https://raw.githubusercontent.com/github/gitignore/main/Node.gitignore}. The latest application files in the \textbf{main} branch will be used to mark against the \textbf{Functionality} criterion. Please test before you submit. Partial marks \textbf{will not} be given for incomplete functionality. Late submissions will incur a \textbf{10\% penalty per day}, rolling over at \textbf{5:00 PM}.

\section*{Extensions}
Familiarise yourself with the assessment due date. Extensions will \textbf{only} be granted if you are unable to complete the assessment by the due date because of \textbf{unforeseen circumstances outside your control}. The length of the extension granted will depend on the circumstances and \textbf{must} be negotiated with the course lecturer before the assessment due date. A medical certificate or support letter may be needed. Extensions will not be granted for poor time management or pressure of other assessments.

\section*{Resits}
Resits and reassessments \textbf{are not} applicable in \textbf{ID608001: Intermediate Application Development Concepts}.

\section*{Instructions}

\subsection*{Functionality - Learning Outcomes 1 and 2 (50\%)}
\begin{itemize} 

\item \textbf{RESTful API:}
	\begin{itemize}
		\item \textbf{Note:} You are required to use the provided \textbf{schema.prisma} file. You can find this in the \textbf{assessments} directory of the \textbf{course materials} repository.
		\item \textbf{User:}
		\begin{itemize}
			\item You will have \textbf{two} types of users - \textbf{super admin} and \textbf{basic} user.
			\item A \textbf{super admin} and \textbf{basic} user will have the following information: first name, last name, username, email address, profile picture, password, confirm password and role. The confirm password does not need to be stored in the database. The users' profile picture will be from the following \textbf{API} - \href{https://avatars.dicebear.com/docs/http-api}{https://avatars.dicebear.com/docs/http-api}. Each profile picture should be, in most cases, different. I suggest using a random seed when setting the user's profile picture.
			\item A \textbf{super admin} and \textbf{basic} user can login, get their information and update their information but not delete themselves. A \textbf{super admin} user can get all users' information, update all \textbf{basic} users' information but not \textbf{super admin} user's information and delete all \textbf{basic} users but not \textbf{super admin} users. A \textbf{basic} user can register.
			\item When performing a \textbf{POST} request for registering a \textbf{basic} user, the following error checking needs to be implemented using \textbf{Joi} and/or conditional statements:
			\begin{itemize}
				\item First name has a minimum length of two characters, a maximum length of 50 characters and alpha characters only.
				\item Last name has the same error checking as first name above.
				\item Username is unique, has a minimum length of five characters, maximum length of ten characters and alphanumeric characters only, i.e., johndoe123.
				\item Email address is unique, contains the username above, an @ special character and a second-level domain, i.e., johndoe123@email.com.
				\item Password has a minimum length of eight characters, maximum length of 16 characters and contains one numeric character and one special character.
				\item Confirm password is the same as the password above. \textbf{Note:} Confirm password will not be a field in the \textbf{User} table. Rather, it will be used to validate the user's password.
			\end{itemize}
			For each error check, a status code and response message is returned, i.e., "First name must have a minimum length of two characters".
			\item When performing a \textbf{POST} request for logging in a user using either username/password or email address/password, return a status code, a response message, i.e., "$<$User's username$>$ has successfully logged in" and the user's \textbf{JWT}.
			\item When performing a \textbf{PUT} and \textbf{DELETE} request, return a status code and a response message, i.e., "$<$User's username$>$'s information has successfully updated" or "$<$User's username$>$ has successfully deleted".
			\item Five \textbf{super admin} users are seeded via a \textbf{script} in the \textbf{package.json} file. The \textbf{super admin} users' data will be fetched from a local file and inserted into the \textbf{User} table using \textbf{Prisma}.  
			\item Five \textbf{basic} users are seeded via a \textbf{super admin} user. The \textbf{basic} users' data will be fetched from a private \textbf{GitHub Gist} using \textbf{Axios} and inserted into the \textbf{User} table using \textbf{Prisma}. 
		\end{itemize} 
		\item \textbf{Quiz:}
		\begin{itemize}
			\item Each quiz will have the following information: name, start date, end date, category, difficulty, type, number of questions, list of questions, list of correct answers, list of incorrect answers, list of scores and average score. The category, list of questions, list of correct answers and list of incorrect answers will be fetched from the following \textbf{API} - \href{https://opentdb.com/api\_config.php}{https://opentdb.com/api\_config.php}. The difficulties will be easy, medium and hard. The types will be multiple choice or true/false.
			A \textbf{super admin} user can create and delete a quiz. A \textbf{basic} user can participate in a quiz only once. A \textbf{super admin}, \textbf{basic} and \textbf{unauthenticated} user can get all quizzes, get all past quizzes, get all present quizzes and get all future quizzes, get a list of scores for a quiz and average score for a quiz.
			\item When performing a \textbf{POST} request for creating a quiz, the following error checking needs to be implemented using \textbf{Joi} and/or conditional statements:
			\begin{itemize}
				\item Name has a minimum length of five characters, a maximum length of 30 characters and alpha characters only.
				\item Start date has to be greater than or equal to today's date.
				\item End date has to be greater than the start date and no longer than five days. 
				\item Number of questions has to be ten.
			\end{itemize}
			For each error check, a status code and response message is returned, i.e., "Name must have a minimum length of five characters".
			\item When performing a \textbf{POST} request for a \textbf{basic} user who is participating in a quiz, the following error checking needs to be implemented using \textbf{Joi} and/or conditional statements:
			\begin{itemize}
				\item Can not participate if quiz has not started or has ended.
				\item Answered all ten questions.
			\end{itemize}
			\item When performing a \textbf{POST} request for a \textbf{basic} user who has participated in a quiz, return a status code, a response message, i.e., "$<$User's username$>$ has successfully participated in $<$Quiz's name$>$", user's score and quiz's average score.
		\end{itemize}
		\item Implement versioning, \textbf{Helmet}, \textbf{CORS}, \textbf{rate limiting}, \textbf{caching} and \textbf{compression}
	\end{itemize} 

	\item \textbf{Frontend Application - Quiz:}
	\begin{itemize}
		\item \textbf{User:}
		\begin{itemize}
			\item Create a page that allows a \textbf{basic} user to register.
			\item Create a page that allows a \textbf{super admin} and \textbf{basic} user to login.
			\item Create pages that allow:
			\begin{itemize}
				\item A \textbf{super admin} user to get their information, update their information but not delete themselves, get all user's information, update all \textbf{basic} users' information but not \textbf{super admin} user's information and delete all \textbf{basic} users but not \textbf{super admin} users.
				\item A \textbf{basic} user to get their information and update their information but not delete themselves.
			\end{itemize}
		\end{itemize}
		\item \textbf{Quiz:} 
		\begin{itemize}
			\item Create pages that allow:
			\begin{itemize}
				\item A \textbf{super admin} user to create and delete a quiz.
				\item A \textbf{basic} user to participate in a quiz only once get all quizzes.
				\item A \textbf{super admin}, \textbf{basic} and \textbf{unauthenticated} user to get all quizzes, get all past quizzes, get all present quizzes, get all future quizzes, list of scores for a quiz and average score for a quiz.
			\end{itemize}
		\end{itemize}
		\item Create a navigation bar that allows a \textbf{super admin}, \textbf{basic} and \textbf{unauthenticated} user to navigate to the pages above.
		\item Incorrectly formatted form field values handled gracefully using validation error messages.
		\item Create a footer that displays the current year and the name of the application.
		\item UI is visually attractive with a coherent graphical theme and style using \textbf{Tailwind CSS} and \textbf{Shadcn UI}.
	\end{itemize}

	\item \textbf{Frontend Application - RESTful API:}
	\begin{itemize}
		\item Create a page that allows an \textbf{unauthenticated} user to get \textbf{RESTful API} information. This includes:
		\begin{itemize}
			\item Name of the endpoint. For example, \textbf{Register Basic User}.
			\item HTTP method. For example, \textbf{POST}.
			\item Endpoint URL. For example, \textbf{/api/v1/auth/register}.
			\item Request body.
			\item Response body.
		\end{itemize}
		\item UI is visually attractive with a coherent graphical theme and style using \textbf{Tailwind CSS} and \textbf{Shadcn UI}.
	\end{itemize}

	\item \textbf{Testing:}
	\begin{itemize}
		\item Implement the following \textbf{end-to-end tests} using \textbf{Cypress}:
		\begin{itemize}
			\item Registering a \textbf{basic} user.
			\item Logging in a \textbf{super admin} user and creating a quiz.
			\item Logging in a \textbf{basic} user and participating in a quiz.
		\end{itemize}
	\end{itemize}

	\item \textbf{Scripts:}
	\begin{itemize}
		\item Run your \textbf{RESTful API} and \textbf{frontend applications} locally.
		\item Run your \textbf{end-to-end tests} using \textbf{Cypress}.
		\item Create and apply a migration using \textbf{Prisma}.
		\item Reset your database using \textbf{Prisma}.
		\item Seed \textbf{super admin} users.
		\item Open \textbf{Prisma Studio}. 
		\item Lint your code using \textbf{ESLint}.
		\item Format your code using \textbf{Prettier}.
	\end{itemize}
\end{itemize}

\subsection*{Code Quality and Best Practices - Learning Outcome 1 (45\%)}
\begin{itemize}
	\item A \textbf{Node.js} \textbf{.gitignore} file is used.
	\item Environment variables' key is stored in the \textbf{.env.example} file. 
  	\item Appropriate naming of files, variables, functions and resource groups.
  \begin{itemize}
	\item Resource groups are named with a plural noun instead of a noun or verb, i.e., \textbf{/api/v1/items} not \textbf{/api/v1/item}.
  \end{itemize}
	\item Idiomatic use of control flow, data structures and in-built functions.
  \item Efficient algorithmic approach.
  \item Sufficient modularity.
  \item Each \textbf{controller}, \textbf{route} and \textbf{component} file has a \textbf{JSDoc} header comment located at the top of the file.
  \item Code is formatted using \textbf{Prettier}.
  \item Code is linted using \textbf{ESLint}.
  \item \textbf{Prettier}, \textbf{ESLint} and \textbf{Cypress} are installed as a \textbf{development dependency}.	
\item No dead or unused code. 
\end{itemize}

\subsection*{Documentation and Git Usage - Learning Outcomes 1 (5\%)}
\begin{itemize}
	\item A \textbf{GitHub} project board to help you organise and prioritise your development work. The course lecturer needs to see consistent use of the \textbf{GitHub} project board for the duration of the assessment.
	\item Provide the following in your repository \textbf{README.md} file:
	\begin{itemize}
		\item A URL to your \textbf{RESTful API} as a \textbf{web service} on \textbf{Render}.
		\item A URL to your published \textbf{RESTful API} documentation. Each route needs to be documented. Include a description, example request and example response.
		\item How do you setup the environments, i.e., after the repository is cloned?
		\item How do you run your \textbf{RESTful API} and \textbf{frontend applications} locally?
		\item How do you run your \textbf{End-to-end tests}?
		\item How do you create and apply a migration?  
		\item How do you reset your database?
		\item How do you seed \textbf{super admin} users?
		\item How do you open \textbf{Prisma Studio}?
		\item How do you format your code?
		\item How do you lint your code?		
	\end{itemize}
    \item Use of \textbf{Markdown}, i.e., headings, bold text, code blocks, etc.
    \item Correct spelling and grammar.
    \item Your \textbf{Git commit messages} should:
    \begin{itemize}
      \item Reflect the context of each functional requirement change.
      \item Be formatted using an appropriate naming convention style.
    \end{itemize}
\end{itemize}

\subsection*{Additional Information}
\begin{itemize}
	\item Exemplars are available:
	\begin{itemize}
		\item \textbf{RESTful API} - \href{https://id608001-graysono-wbnj.onrender.com}{https://id608001-graysono-wbnj.onrender.com}
		\item \textbf{Frontend Applications} - In \textbf{assessments} directory of the \textbf{course materials} repository.
	\end{itemize}
    \item You need to show the course lecturer the initial \textbf{GitHub} project board before you start your development work. Following this, you need to show the course lecturer your \textbf{GitHub} project board at the end of each week.
    \item \textbf{Do not} rewrite your \textbf{Git} history. It is important that the course lecturer can see how you worked on your assessment over time. 
\end{itemize} 
\end{document}
\end{document}
