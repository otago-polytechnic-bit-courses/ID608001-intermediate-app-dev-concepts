% Author: Grayson Orr
% Course: ID737001: Game Development

\documentclass{article}
\author{}

\usepackage{graphicx}
\usepackage{wrapfig}
\usepackage{enumerate}
\usepackage{hyperref}
\usepackage[margin = 2.25cm]{geometry}
\usepackage[table]{xcolor}
\usepackage{fancyhdr}
\hypersetup{
  colorlinks = true,
  urlcolor = blue
}
\setlength\parindent{0pt}
\pagestyle{fancy}
\fancyhf{}
\rhead{College of Engineering, Construction and Living Sciences\\Bachelor of Information Technology}
\lfoot{Client Project\\Version 2, Semester One, 2023}
\rfoot{\thepage}
 
\begin{document} 

\begin{figure}
	\centering
	\includegraphics[width=50mm]{../../resources/img/logo.png}
\end{figure}

\title{College of Engineering, Construction and Living Sciences\\Bachelor of Information Technology\\ID737001: Game Development\\Level 7, Credits 15\\\textbf{Client Project}}
\date{}
\maketitle

\section*{Assessment Overview}
In this \textbf{individual} assessment, you will work in a team of \textbf{BIT} \& \textbf{BDes} learners to develop a game in \textbf{Unity}. You will be assessed on your ability to demonstrate technical \& professional proficiency, \& robust documentation \& reflective practices.

\section*{Learning Outcomes}
At the successful completion of this course, learners will be able to:
\begin{enumerate}
	\item Design \& develop a game using industry standard tools, technologies \& practices.
\end{enumerate}

\section*{Assessments}
\renewcommand{\arraystretch}{1.5}
\begin{tabular}{|c|c|c|c|}
	\hline
	\textbf{Assessment}                                 & \textbf{Weighting} & \textbf{Due Date}            & \textbf{Learning Outcomes} \\ \hline
	\small Game Jam  & \small 30\%        & \small TBC   & \small 1                   \\ \hline
	\small Client Project & \small 70\%        & \small 13-06-2023 (Tuesday at 4.59 PM)   & \small 1                   \\ \hline
\end{tabular} 

\section*{Conditions of Assessment}
You will complete this assessment during your learner managed time, however, there will be availability during the weekly meetings to discuss the requirements \& your progress of this assessment. This assessment will need to be completed by \textbf{Tuesday, 13 June 2023} at \textbf{4.59 PM}.

\section*{Pass Criteria}
This assessment is criterion-referenced (CRA) with a cumulative pass mark of \textbf{50\%} over all assessments in \textbf{ID737001: Game Development}.

\section*{Authenticity}
All parts of your submitted assessment \textbf{must} be completely your work \& any references \textbf{must} be cited appropriately including, externally-sourced graphic elements. Failure to do this will result in a mark of \textbf{zero} for this assessment.

\section*{Policy on Submissions, Extensions, Resubmissions \& Resits}
The school's process concerning submissions, extensions, resubmissions \& resits complies with \textbf{Otago Polytechnic} policies. Learners can view policies on the \textbf{Otago Polytechnic} website located at \href{https://www.op.ac.nz/about-us/governance-and-management/policies}{https://www.op.ac.nz/about-us/governance-and-management/policies}.

\section*{Submission}
You \textbf{must} submit your report via email. Late submissions will incur a \textbf{10\% penalty per day}, rolling over at \textbf{5:00 PM}.

\section*{Extensions}
Familiarise yourself with the assessment due date. If you need an extension, contact the course lecturer(s) before the due date. If you require more than a week's extension, a medical certificate or support letter from your manager may be needed.

\section*{Resubmissions}
Learners may be requested to resubmit an assessment following a rework of part/s of the original assessment. Resubmissions are to be completed within a negotiable short time frame \& usually \textbf{must} be completed within the timing of the course to which the assessment relates. Resubmissions will be available to learners who have made a genuine attempt at the first assessment opportunity \& achieved a \textbf{D grade (40-49\%)}. The maximum grade awarded for resubmission will be \textbf{C-}.

\section*{Resits}
Resits \& reassessments \textbf{are not} applicable in \textbf{ID737001: Game Development}.

\section*{Instructions}

\subsection*{Technical \& Professional Proficiency - Learning Outcomes 1 (80\%)}
\begin{itemize}
	\item Identify the client requirements and deconstruct them into technical tasks. Also, track the progress of each task using an appropriate development methodology, i.e., \textbf{Kanban} \& platform, i.e., \textbf{GitHub} project \& issues boards. 
	\item Contribute a meaningful amount of code to the client project. Also, perform other technical jobs such as code reviews \& user testing. 
	\item Communicate within your team, i.e., \textbf{BIT} \& \textbf{BDes} learners, to maintain sustainable productivity. Forms of communication include but are not limited to face-to-face or online team meetings, instant messaging \& in-class discussions. 
	\item Use industry-standard tools/platforms for communication, i.e., \textbf{Microsoft Teams}, \textbf{Discord} \& project management tools, i.e., \textbf{GitHub} \&/or \textbf{Trello}, professionally. 
	\item \textbf{Git} commit messages reflect the context of each functional requirement change.
	\item Code elegance:
	\begin{itemize}
		\item Idiomatic use of the game engine, i.e., specific features in \textbf{Unity}.
		\item Sufficient modularity, i.e., mechanics are distinct and separate units of functionality.
	\end{itemize}
\end{itemize}

\subsection*{Documentation - Learning Outcomes 1 (20\%)}
In a \textbf{Microsoft Word} document, include the following:
\begin{itemize}
	\item Decisions related to technologies and techniques, i.e., \textbf{Unity}, \textbf{GitHub}, etc., based on evidence. \textbf{Note:} Where multiple solutions are possible, please show evidence of evaluation that has led you to a decision.
	\item Select an interesting game mechanic that \textbf{you} implemented \& describe the following:
	\begin{itemize}
		\item What did you implement?
		\item What did you research during the implementation?
		\item What did you try? What worked? What did not work?
		\item What did you learn?
		\item How can you apply what you learned to other contexts, i.e., another game you may develop in the future?
	\end{itemize}
\end{itemize}
\textbf{Note:} This document \textbf{must} be submitted via email. 

\subsection*{Additional Information}
\begin{itemize}
	\item Add your course lecturer(s) to your \textbf{GitHub} repository.
	\item Attempt to commit at least \textbf{ten} times per week.
	\item \textbf{Do not} rewrite your \textbf{Git} history. It is important that the course lecturer(s) can see how you worked on your assessment over time.
	\item All \textbf{Git} commit messages must identify which member(s) participated in the associated work session. The proportional contribution will be determined by inspection of the commit logs. Suppose the commit logs show evidence of significantly uneven contribution proportion. In that case, the course lecturer(s) may choose to adjust the mark of the lesser contributor downward by an amount derived from the individual contributions. 
\end{itemize}

\end{document}