\documentclass{article}
\author{}

\usepackage{graphicx}
\usepackage{wrapfig}
\usepackage{enumerate}
\usepackage{hyperref}
\usepackage{float}
\usepackage{color, soul}
\usepackage{alltt}
\usepackage[margin = 2.25cm]{geometry}
\usepackage[table]{xcolor}
\usepackage{fancyhdr}
\hypersetup{
  colorlinks = true,
  urlcolor = blue
}
\setlength\parindent{0pt}
\pagestyle{fancy}
\fancyhf{}
\rhead{College of Engineering, Construction and Living Sciences\\Bachelor of Information Technology}
\lfoot{Practical 21 React 6: Forms \\Version 2, 2021}
\rfoot{\thepage}

\begin{document}

\begin{figure}
	\centering
	\includegraphics[width=50mm]{img/logo.png}
\end{figure}

\title{College of Engineering, Construction and Living Sciences\\Bachelor of Information Technology\\IN608: Intermediate Application Development Concepts\\Level 6, Credits 15\\\textbf{Practical 21 React 6: Forms}} 
\date{}
\maketitle

\textbf{Due Date:} 23-06-2021 at 5pm \\

In this practical, you will complete a series of tasks covering today's lecture. This practical is worth 1\% of the final mark for the IN608: Intermediate Application Development Concepts course. \\

Before you start, in your practicals repository, create a new branch called \textbf{21-practical}. \\

\section*{Task 1} 
Create a Django project called \texttt{backend}. \texttt{cd} to \texttt{backend}, create a virtual environment \& install \textbf{Django}. Create an app called \texttt{practical21backend}. Please ensure you configure your app in \texttt{backend/settings.py}.

\begin{alltt}
  # backend/settings.py

  INSTALLED_APPS = [
      \hl{\textbf{'practical21backend',}} 
      'django.contrib.admin',
      'django.contrib.auth',
      'django.contrib.contenttypes',
      'django.contrib.sessions',
      'django.contrib.messages',
      'django.contrib.staticfiles',
  ]
\end{alltt}

Create a model to define how the \texttt{Todo} items should be stored in the database, open the \texttt{practical21backend/models.py} file \& update it with the following:

\begin{verbatim}
  # practical21backend/models.py

  from django.db import models

  class Todo(models.Model):
      title = models.CharField(max_length=120)
      description = models.TextField()
      completed = models.BooleanField(default=False)

  def __str__(self):
      return self.title
\end{verbatim}

We have created a \texttt{Todo} model, we need to create a migration file \& apply the changes to the database, so let’s run these commands:

\begin{verbatim}
  python manage.py makemigrations
  python manage.py migrate 
\end{verbatim}

Install the \texttt{djangorestframework} \& \texttt{django-cors-headers} using \texttt{Pipenv}:

\begin{verbatim}
  pipenv install djangorestframework django-cors-headers
\end{verbatim}

We need to add \texttt{rest\_framework} \& \texttt{corsheaders} to the list of installed applications, so update the \texttt{INSTALLED\_APPS} \& \texttt{MIDDLEWARE} sections in \texttt{backend/settings.py}:

\begin{alltt}
  # backend/settings.py

  INSTALLED_APPS = [
      'practical21backend',
      'django.contrib.admin',
      'django.contrib.auth',
      'django.contrib.contenttypes',
      'django.contrib.sessions',
      'django.contrib.messages',
      'django.contrib.staticfiles',
      \hl{\textbf{'corsheaders',}}                          
      \hl{\textbf{'rest_framework',}}          
  ]

  MIDDLEWARE = [
      \hl{\textbf{'corsheaders.middleware.CorsMiddleware',}} 
      'django.middleware.security.SecurityMiddleware',
      'django.contrib.sessions.middleware.SessionMiddleware',
      'django.middleware.common.CommonMiddleware',
      'django.middleware.csrf.CsrfViewMiddleware',
      'django.contrib.auth.middleware.AuthenticationMiddleware',
      'django.contrib.messages.middleware.MessageMiddleware',
      'django.middleware.clickjacking.XFrameOptionsMiddleware',
  ]

  CORS_ORIGIN_WHITELIST = [
    'http://localhost:3000'
  ]

\end{alltt}

\texttt{django-cors-headers} is a python library that will help in preventing the errors that we would normally get due to \texttt{CORS}. rules. In the \texttt{CORS\_ORIGIN\_WHITELIST} snippet, we whitelisted \texttt{http://localhost:3000/} because we want the React application to interact with the Djnago REST API. \\

Create a new \texttt{practical21backend/serializers.py} file \& update it with the following code:

\begin{verbatim}
  from rest_framework import serializers
  from .models import Todo

  class TodoSerializer(serializers.ModelSerializer):
      class Meta:
          model = Todo
          fields = ['id', 'title', 'description', 'completed']
\end{verbatim}

In the code snippet above, we specified the model to work with \& the fields we want to be converted to \textbf{JSON}. \\

Open \texttt{practical21backend/views.py} file \& update it with the following code:

\begin{verbatim}
  from django.shortcuts import render
  from rest_framework import viewsets          
  from .serializers import TodoSerializer      
  from .models import Todo                     

  class TodoView(viewsets.ModelViewSet):       
      serializer_class = TodoSerializer          
      queryset = Todo.objects.all()              
\end{verbatim}

The \texttt{viewsets} base class provides the implementation for \textbf{CRUD} operations by default, what we had to do was specify the serializer class \& the \texttt{queryset}. \\

Open \texttt{backend/urls.py} file \& update it with the following code:

\begin{verbatim}
  from django.contrib import admin
  from django.urls import path, include                
  from rest_framework import routers                   
  from practical21backend import views                            

  router = routers.DefaultRouter()                  
  router.register(r'todos', views.TodoView, 'todo') 

  urlpatterns = [
      path('admin/', admin.site.urls),         
      path('api/', include(router.urls))               
  ]
\end{verbatim}

This is the final step that completes the building of the \textbf{API}, we can now perform \textbf{CRUD} operations on the \texttt{Todo} model. 

\section*{Task 2} 

Create a new \textbf{React} application called \texttt{frontend}.

\begin{verbatim}
  npx create-react-app frontend
\end{verbatim}

Install the following packages:

\begin{verbatim}
  npm install bootstrap reactstrap axios
\end{verbatim}

In \texttt{src/index.js}, declare \texttt{import 'bootstrap/dist/css/bootstrap.min.css'}. This will allow you to use \textbf{Bootstrap} in your application. \\

Open the \texttt{src/index.css} file \& replace the styles with the following:

\begin{verbatim}
  body {
    margin: 0;
    padding: 0;
    font-family: -apple-system, BlinkMacSystemFont, "Segoe UI", "Roboto", "Oxygen",
      "Ubuntu", "Cantarell", "Fira Sans", "Droid Sans", "Helvetica Neue",
      sans-serif;
    -webkit-font-smoothing: antialiased;
    -moz-osx-font-smoothing: grayscale;
  }

  .todo-title {
    cursor: pointer;
  }

  .completed-todo {
    text-decoration: line-through;
  }

  .tab-list > span {
    padding: 5px 8px;
    border: 1px solid #000000;
    margin-right: 5px;
    cursor: pointer;
  }

  .tab-list > span.active {
    background-color: #000000;
    color: #ffffff;
  }
\end{verbatim}

Create a new directory called \texttt{components}. In the \texttt{components} directory, copy \texttt{Modal.js} which is given to you in the \texttt{21-react-6-forms} directory. Replace \texttt{App.js} with the \texttt{App.js} file given to you in the same directory. Comments have been provided in these two files. \\

In package.json, add \texttt{"proxy": "http://localhost:8000"} above the \texttt{"dependencies"}. The proxy will help in tunnelling API requests to \texttt{http://localhost:8000} where the \texttt{Django} application will handle them. \\

Once you have done this, run both applications:

\begin{verbatim}
  python manage.py runserver
  npm start
\end{verbatim}

\section*{Task 3} 
\begin{itemize}
  \item Django application
  \begin{itemize}
    \item Create five unit tests covering the \texttt{Todo} model, \texttt{TodoSerializer} serializer \& API
  \end{itemize}
  \item React application
  \begin{itemize}
    \item Convert the \textbf{CSS} in \texttt{src/index.css} to \textbf{Sass}
    \item Ensure the \texttt{title} \& \texttt{description} form inputs are required
    \item Display an appropriate message, if there are no \texttt{Todo} items in the list
  \end{itemize}
\end{itemize} 

\section*{Task 4} 
Please answer the following question in the comment section once you make a pull request:
\begin{itemize}
  \item Describe three reasons why an application's frontend \& backend should be separated.
\end{itemize} 

\subsection*{Resources} 
\begin{itemize}
  \item \href{https://developer.mozilla.org/en-US/docs/Web/HTTP/CORS}{CORS}
\end{itemize}
 
\end{document}