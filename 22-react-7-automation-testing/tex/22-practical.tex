\documentclass{article}
\author{}

\usepackage{graphicx}
\usepackage{wrapfig}
\usepackage{enumerate}
\usepackage{hyperref}
\usepackage{float}
\usepackage{color, soul}
\usepackage{alltt}
\usepackage[margin = 2.25cm]{geometry}
\usepackage[table]{xcolor}
\usepackage{fancyhdr}
\hypersetup{
  colorlinks = true,
  urlcolor = blue
}
\setlength\parindent{0pt}
\pagestyle{fancy}
\fancyhf{}
\rhead{College of Engineering, Construction and Living Sciences\\Bachelor of Information Technology}
\lfoot{Practical 22 React 7: Automation Testing \\Version 2, 2021}
\rfoot{\thepage}

\begin{document}

\begin{figure}
	\centering
	\includegraphics[width=50mm]{img/logo.png}
\end{figure}

\title{College of Engineering, Construction and Living Sciences\\Bachelor of Information Technology\\IN608: Intermediate Application Development Concepts\\Level 6, Credits 15\\\textbf{Practical 22 React 7: Automation Testing}} 
\date{}
\maketitle

\textbf{Due Date:} 23-06-2021 at 5pm \\

In this \textbf{self-directed} practical, you will complete a series of tasks covering today's lecture. This practical is worth 1\% of the final mark for the IN608: Intermediate Application Development Concepts course. \\

Before you start, in your practicals repository, create a new branch called \textbf{22-practical}. 

\section*{Task} 
Please use the completed practical from \texttt{Practical 21 React 6: Forms}. In the \texttt{frontend} directory, create a file called \texttt{App.test.js}. You are going to use \textbf{Cypress.IO} \& the resources to test the following:

\begin{itemize}
  \item Form interactions
  \begin{itemize}
    \item Create a new \texttt{Todo} item
    \item Delete a \texttt{Todo} item
    \item Update a \texttt{Todo} item
    \item Resource: \footnotesize\href{https://github.com/cypress-io/cypress-example-recipes/tree/master/examples/blogs\_\_e2e-api-testing}{https://github.com/cypress-io/cypress-example-recipes/tree/master/examples/blogs\_\_e2e-api-testing}
  \end{itemize}
  \item API Testing
  \begin{itemize}
    \item \texttt{GET /api/todos/} returns status code 200
    \item \texttt{GET /api/todos/<ITEM\_ID>/} returns status code 200
    \item Resource: \footnotesize\href{https://github.com/cypress-io/cypress-example-recipes/tree/master/examples/testing-dom\_\_form-interactions}{https://github.com/cypress-io/cypress-example-recipes/tree/master/examples/testing-dom\_\_form-interactions}
  \end{itemize}
\end{itemize}

To use \textbf{Cypress.IO}, you must install the following package:

\begin{verbatim}
  npm i cypress
\end{verbatim}

\textbf{Note:} Make sure you have your \texttt{backend} \textbf{Django} application running in the background. Remember, this is contains all your endpoints in which \textbf{Cypress.IO} needs to access.

\subsection*{Resources} 
\begin{itemize}
  \item \href{https://www.npmjs.com/package/cypress}{Cypress.IO npm}
  \item Cypress.IO resources:
  \begin{itemize}
    \item \href{https://docs.cypress.io/examples/examples/recipes.html#Fundamentals}{Cypress.IO Recipes}
    \item \href{https://docs.cypress.io/examples/examples/recipes.html#Fundamentals}{Cypress.IO Tutorial Videos}
  \end{itemize}
\end{itemize}
 
\end{document}