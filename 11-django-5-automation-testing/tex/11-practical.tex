\documentclass{article}
\author{}

\usepackage{graphicx}
\usepackage{wrapfig}
\usepackage{enumerate}
\usepackage{hyperref}
\usepackage{float}
\usepackage[margin = 2.25cm]{geometry}
\usepackage[table]{xcolor}
\usepackage{fancyhdr}
\hypersetup{
  colorlinks = true,
  urlcolor = blue 
}
\setlength\parindent{0pt}
\pagestyle{fancy}
\fancyhf{}
\rhead{College of Engineering, Construction and Living Sciences\\Bachelor of Information Technology}
\lfoot{Practical 11 Django 5: Automation Testing \\Version 2, 2021}
\rfoot{\thepage}

\begin{document}

\begin{figure}
	\centering
	\includegraphics[width=50mm]{./img/logo.png}
\end{figure}

\title{College of Engineering, Construction and Living Sciences\\Bachelor of Information Technology\\IN608: Intermediate Application Development Concepts\\Level 6, Credits 15\\\textbf{Practical 11 Django 5: Automation Testing}} 
\date{}
\maketitle

\textbf{Due Date:} 07-05-2021 at 5pm \\

In this practical, you will complete a series of tasks covering today's lecture. This practical is worth 1\% of the final mark for the IN608: Intermediate Application Development Concepts course. \\

Before you start, in your practicals repository, create a new branch called \textbf{11-practical}.

\section*{Task 1} 
Copy \& paste the \texttt{dog} Django project from \textbf{10-practical}. Run the virtual environment by running the command \texttt{pipenv shell}. In \texttt{tests.py}, create three \texttt{TestCase} classes called \texttt{TestDog}, \texttt{TestIndexView} \& \texttt{TestResults}.  \\

\texttt{TestDog} will contain the following tests:
\begin{itemize}
  \item \texttt{Dog} object's \texttt{breed} equals some value
  \item \texttt{Dog} object's \texttt{height} equals some value
  \item \texttt{Dog} object's \texttt{adaptability} equals \texttt{High}
  \item Field's \texttt{max\_length} equals some value
\end{itemize}

\texttt{TestIndexView} will contain the following tests:
\begin{itemize}
  \item Status code equals 200
  \item Template used is \texttt{practical10cdns/index.html}
\end{itemize}

\texttt{TestResults} will contain the following tests:
\begin{itemize}
  \item Template used is \texttt{practical10cdns/results.html}
  \item QuerySet returns no \texttt{Dog} objects
  \item QuerySet returns one or more \texttt{Dog} objects
\end{itemize}

Create a \texttt{StaticLiveServerTestCase} called \texttt{TestResultSelenium}. Install selenium by running the command \texttt{pipenv install selenium}. Check the \texttt{Pipfile} \& \texttt{Pipfile.lock} to ensure it has install successfully. 

\texttt{TestResultSelenium} will contain the following tests:
\begin{itemize}
  \item The text \textbf{1 dog matched your search.} is found in the page source
  \item The text \textbf{2 dogs matched your search.}  is found in the page source
  \item The text \textbf{No dogs available.} is found in the page source
\end{itemize}

Run the command \texttt{python manage.py tests} to run the tests.

\section*{Task 2} 
Use Coverage.py to measure the code coverage of the project. Coverage.py analyses a program to identify code that was executed \& code that was not executed, but could be executed. Coverage measurement is an excellent way to gauge the effectiveness of tests. To install \texttt{Coverage.py}, run the command \texttt{pipenv install coverage}. Generate a coverage report of the Django project in a nice HTML format. 

\subsection*{Resources} 
\begin{itemize}
  \item \href{https://selenium-python.readthedocs.io/}{Selenium with Python}
  \item \href{https://coverage.readthedocs.io/en/coverage-5.2.1/}{Coverage.py}
\end{itemize}

\end{document}