% Author: Grayson Orr
% Course: IN608: Intermediate Application Development Concepts

\documentclass{article}
\author{}

\usepackage{graphicx}
\usepackage{wrapfig}
\usepackage{enumerate}
\usepackage{hyperref}
\usepackage[margin = 2.25cm]{geometry}
\usepackage[table]{xcolor}
\usepackage{fancyhdr}
\hypersetup{
  colorlinks = true,
  urlcolor = blue
}
\setlength\parindent{0pt}
\pagestyle{fancy}
\fancyhf{}
\rhead{College of Engineering, Construction and Living Sciences\\Bachelor of Information Technology}
\lfoot{Django \& OpenTDB API \\Version 1, 2020}
\rfoot{\thepage}

\begin{document}

\begin{figure}
	\centering
	\includegraphics[width=50mm]{./img/logo.png}
\end{figure}

\title{College of Engineering, Construction and Living Sciences\\Bachelor of Information Technology\\IN608: Intermediate Application Development Concepts\\Level 6, Credits 15\\\textbf{Django \& OpenTDB API}}
\date{}
\maketitle

\section*{Assessment Overview}
For this assessment, you will design, develop \& deploy a quiz tournament application using Django, the OpenTDB API \& Heroku. The main purpose of this assessment is not just to build a full-stack application, rather to demonstrate sound programming by following the Model, View, Template architectural design pattern \& SOLID principles. Marks will be allocated for functionality \& best practices such as application robustness, code elegance, documentation \& git usage. \\

Due to a nation-wide lockdown, your local pub is no longer able to run their weekly quiz tournament onsite. Your local pub owners know you are an IT student \& ask if you want create an online quiz tournament application for them. The pub owners want an application that allows users to signup, login, participate in various quiz tournaments \& keep track of scores so that they can give away prizes at the end of each quiz tournament. In addition to the basic features, you suggest a variety of features which will enhance the quiz experience.

\section*{Assessment Table}
\renewcommand{\arraystretch}{1.5}
\begin{tabular}{|l|l|l|l|l|}
	\hline		
	\vtop{\hbox{\strut \textbf{Assessment}}\hbox{\strut \textbf{Activity}}} & \textbf{Weighting} & \vtop{\hbox{\strut \textbf{Learning}}\hbox{\strut \textbf{Outcomes}}} & \vtop{\hbox{\strut \textbf{Assessment}}\hbox{\strut \textbf{Grading Scheme}}} & \vtop{\hbox{\strut \textbf{Completion}}\hbox{\strut \textbf{Requirements}}} \\
					
	\hline
						
	\small Practicals                                                       & \small 20\%        & \small 1                                                              & \small CRA                                                                    & \small Cumulative                                                           \\ \hline
	\small Django \& OpenTDB API                                            & \small 50\%        & \small 1, 2                                                           & \small CRA                                                                    & \small Cumulative                                                           \\ \hline
	\small Django REST Framework, React \& OpenTDB API                      & \small 30\%        & \small 1, 2                                                           & \small CRA                                                                    & \small Cumulative                                                           \\ \hline   
\end{tabular}

\section*{Conditions of Assessment}
This assessment will need to be completed by Friday, 16 October 2020 at 5pm. 

\section*{Pass Criteria}
This assessment is criterion-referenced with a cumulative pass mark of 50\%.

\section*{Submission Details}
You must submit your program files via \textbf{GitHub Classroom}. Here is the link to the repository you will be using for your submission – \href{https://classroom.github.com/a/uQeihzqX}{https://classroom.github.com/a/uQeihzqX}.

\section*{Authenticity}
All parts of your submitted assessment must be completely your work and any references must be cited appropriately.

\section*{Policy on Submissions, Extensions, Resubmissions \& Resits}
The school's process concerning \textbf{Submissions, Extensions, Resubmissions and Resits} complies with Otago Polytechnic policies. Students can view policies on the Otago Polytechnic website located at \href{https://www.op.ac.nz/about-us/governance-and-management/policies}{https://www.op.ac.nz/about-us/governance-and-management/policies}.

\section*{Extensions}
Please familiarise yourself with the assessment due dates. If you need an extension, please contact your lecturer before the due date. If you require more than a week's extension, a medical certificate or support letter from your manager may be needed.

\section*{Resubmissions}
Students may be requested to resubmit an assessment following a rework of part/s of the original assessment. Resubmissions are completed within a short time frame (usually no more than 5 working days) and usually must be completed within the timing of the course to which the assessment relates. Resubmissions will be available to students who have made a genuine attempt at the first assessment opportunity. The maximum grade awarded for resubmission will be C-.

\section*{Learning Outcomes}
At the successful completion of this course, students will be able to:
\begin{enumerate}
	\item Demonstrate sound programming by following design patterns and best practices.
	\item Design and implement full-stack applications using industry relevant programming languages.
\end{enumerate}

\newpage

\section*{Instructions} 
\subsection*{Functionality \& Robustness - Learning Outcomes 1, 2}
\begin{itemize}
	\item Dependencies are correctly managed using \textbf{Pipenv} \& \textbf{Pipfile}.
	\item User (applies to both admin \& player users) features:
	\begin{itemize}
		\item Log into the application using an email/username \& password.
		\item Logout of the application.
		\item Incorrect formatted input values handled gracefully using validation error messages, for example, username input field is blank.
		\item Update user profile using an HTML form. Updatable fields include username, first name, last name, email \& profile picture.
		\item View scores for each quiz tournament. Display the total players participated, average score, number of likes, player's name, completed date \& player's score. Sort the scores in descending order by player’s score.
		\item Request a password reset if user's password is forgotten. If an email exists, the user will be emailed instructions for resetting their password.
	\end{itemize}
	\item Admin user specific features:
	\begin{itemize}
		\item Create a new admin user using \textbf{python manage.py createsuperuser}. Fields must be username, first name, last name \& email. Email must be unique. By default, the admin user's profile picture is set to a placeholder image using a signal. For ease of marking, please create an admin user with the username: \textbf{admin} \& password: \textbf{P@ssw0rd123}
		\begin{itemize}
			\item \textbf{Resource:} \href{https://docs.djangoproject.com/en/3.1/topics/signals}{Django Signals}
		\end{itemize}
		\item Create a new quiz tournament. Fields include name, category, difficulty, start date \& end date. 
		\item A quiz tournament consists of 10 questions dynamically fetched from the \textbf{OpenTDB API}. Categories must be dynamically fetched from the following URL \href{https://opentdb.com/api\_category.php}{https://opentdb.com/api\_category.php}. Choices can not be hardcoded. Questions can be multi-choice, true/false or a mixture of both. Handle gracefully if there are no questions in the response.
		\item Send an email notification to all users when a new quiz tournament has been created. Exclude the admin user who created the quiz tournament.
		\begin{itemize}
			\item \textbf{Resource:} \href{https://docs.djangoproject.com/en/3.1/topics/email}{Django Sending Email}
		\end{itemize}
		\item View all quiz tournaments in an HTML table.
		\item Update a quiz tournament. Updatable fields include name, start date \& end date.
		\item Delete a quiz tournament. Prompt the user for deletion.
		\item View the number of likes for each quiz tournament. Display this information in an appropriate chart, i.e., bar or pie chart using \textbf{Chart.js}.
		\begin{itemize}
			\item \textbf{Resource:} \href{https://www.chartjs.org}{Chart.js}
		\end{itemize}
		\item For each quiz tournament, download a PDF containing a formatted table of scores using \textbf{ReportLab}. Only display the download button if a quiz tournament has one or more scores.
		\begin{itemize}
			\item \textbf{Resource:} \href{https://docs.djangoproject.com/en/3.1/howto/outputting-pdf}{Django Outputting PDFs}
		\end{itemize}
	\end{itemize}
	\item Player user specific features:
	\begin{itemize}
		\item Create a new player user using Django's user authentication system \& an HTML form. Fields \& constraints are the same as the admin user.
		\item Display ongoing, upcoming, past \& participated quiz tournaments. Paginate data across several pages with \textbf{Next/Previous} links.
		\begin{itemize}
			\item \textbf{Resource:} \href{https://docs.djangoproject.com/en/3.1/topics/pagination/}{Django Pagination}
		\end{itemize}
		\item Player user should not be able to participate in upcoming, past or participated quiz tournaments.
		\item Participate in ongoing quiz tournaments. All player users that enter the same quiz tournament will be presented with the same 10 questions.
		\item Questions must presented on separate pages.
		\item Display appropriate feedback for correct \& incorrect answers. If a question is answered incorrectly, display the correct answer.
		\begin{itemize}
			\item \textbf{Resource:} \href{https://docs.djangoproject.com/en/3.1/ref/contrib/messages}{Django Messages}
		\end{itemize}
		\item Player user should be able to leave an ongoing quiz tournament at any time \& return to the last presented question. For example, a player user answers the first five questions then logs out of the application. The player user returns to the quiz tournament three hours later \& is presented with question six.
		\item When the player user's quiz tournament is completed, display their score out of 10.
		\item Like \& unlike quiz tournaments.
	\end{itemize}
	\item Display error pages for the following HTTP status codes:
	\begin{itemize}
		\item 400 - Bad Request
		\item 403 - Forbidden
		\item 404 - Not Found
		\item 500 - Internal Server Error
		\item \textbf{Resource:} \href{https://docs.djangoproject.com/en/3.1/ref/views/#error-views}{Django Error Views}
	\end{itemize}
	\item Visually attractive \& responsive user-interface with a coherent graphical theme \& style using \textbf{Bootstrap} or \textbf{TailWind CSS}.
	\item 
	\begin{itemize}
		\item \textbf{Resources:}
		\begin{itemize}
			\item \href{https://getbootstrap.com}{Bootstrap}
			\item \href{https://tailwindcss.com}{TailWind CSS}
		\end{itemize}
	\end{itemize}
	\item Application deployed to \textbf{Heroku} with \textbf{Gunicorn}.
	\begin{itemize}
		\item \textbf{Resources:} 
		\begin{itemize}
			\item \href{https://devcenter.heroku.com/articles/deploying-python}{Deploying Python and Django Apps on Heroku}
			\item \href{https://gunicorn.org/}{Gunicorn}
		\end{itemize}
	\end{itemize}
	\item Data is persistently stored in \textbf{Heroku PostgreSQL}.
	\begin{itemize}
		\item \textbf{Resource:} \href{https://www.heroku.com/postgres}{Heroku PostgreSQL}
	\end{itemize}
	\item Unit \& integration tests cover models, views, forms \& API.
	\item End-to-end tests cover signup, login, logout, creating, updating \& deleting a quiz tournament \& participating in a quiz tournament.
\end{itemize}

\subsection*{Documentation \& Git Usage - Learning Outcome 1}
\begin{itemize}
    \item Provide the following information in the repository \textbf{README} file:
    \begin{itemize}
		\item How do you set up the environment for development, i.e., after the repository is cloned, what do I need to start coding?
		\item How to run tests.
		\item How to deploy the application.
    \end{itemize}
    \item At least 15 feature branches excluding the \textbf{master} branch.
    \begin{itemize}
        \item Your branches must be prefix with feature}, for example, \textbf{feature-$<$name of functional requirement$>$}.
        \item For each branch, merge your own pull request to the \textbf{master} branch.
    \end{itemize}
    \item Commit messages must reflect the context of each functional requirement change.
\end{itemize}

\end{document}