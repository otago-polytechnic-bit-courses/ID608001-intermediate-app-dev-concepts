% Author: Grayson Orr
% Course: IN608: Intermediate Application Development Concepts

\documentclass{article}
\author{}

\usepackage{graphicx}
\usepackage{wrapfig}
\usepackage{enumerate} 
\usepackage{hyperref} 
\usepackage{float}
\usepackage[margin = 2.25cm]{geometry}
\usepackage[table]{xcolor}
\usepackage{fancyhdr}
\usepackage{makecell}
\usepackage{adjustbox}
\usepackage{pdflscape}
\renewcommand\cellgape{\Gape[3.5pt]}

\hypersetup{
  colorlinks = true,
  urlcolor = blue
}
\setlength\parindent{0pt}
\pagestyle{fancy}
\fancyhf{}
\rhead{College of Engineering, Construction and Living Sciences\\Bachelor of Information Technology}
\lfoot{Practicals \\Version 1, 2020}
\rfoot{\thepage}

\begin{document}

\begin{figure}
	\centering
	\includegraphics[width=50mm]{./img/logo.png}
\end{figure}

\title{College of Engineering, Construction and Living Sciences\\Bachelor of Information Technology\\IN608: Intermediate Application Development Concepts\\Level 6, Credits 15\\\textbf{Practicals}}
\date{}
\maketitle

\section*{Assessment Table}
\renewcommand{\arraystretch}{1.5}
\begin{tabular}{|l|l|l|l|l|}
	\hline		
	\vtop{\hbox{\strut \textbf{Assessment}}\hbox{\strut \textbf{Activity}}} & \textbf{Weighting} & \vtop{\hbox{\strut \textbf{Learning}}\hbox{\strut \textbf{Outcomes}}} & \vtop{\hbox{\strut \textbf{Assessment}}\hbox{\strut \textbf{Grading Scheme}}} & \vtop{\hbox{\strut \textbf{Completion}}\hbox{\strut \textbf{Requirements}}} \\
					
	\hline
						
	\small Practicals                                                       & \small 20\%        & \small 1                                                              & \small CRA                                                                    & \small Cumulative                                                           \\ \hline
	\small Django \& OpenTDB API                                            & \small 50\%        & \small 1, 2                                                           & \small CRA                                                                    & \small Cumulative                                                           \\ \hline
	\small Django REST Framework, React \& OpenTDB API                      & \small 30\%        & \small 1, 2                                                           & \small CRA                                                                    & \small Cumulative                                                           \\ \hline   
\end{tabular}

\section*{Conditions of Assessment}
This assessment will need to be completed by Monday, 2 November 2020 at 5pm.

\section*{Pass Criteria}
This assessment is criterion-referenced with a cumulative pass mark of 50\%.

\section*{Submission Details}
You must submit your program files via \textbf{GitHub Classroom}. Here is the link to the repository you will be using for your submission – \href{https://classroom.github.com/a/2Hnb0QIq}{https://classroom.github.com/a/2Hnb0QIq}.

\section*{Authenticity}
All parts of your submitted assessment must be completely your work and any references must be cited appropriately.

\section*{Policy on Submissions, Extensions, Resubmissions \& Resits}
The school's process concerning \textbf{Submissions, Extensions, Resubmissions and Resits} complies with Otago Polytechnic policies. Students can view policies on the Otago Polytechnic website located at \href{https://www.op.ac.nz/about-us/governance-and-management/policies}{https://www.op.ac.nz/about-us/governance-and-management/policies}.

\section*{Extensions}
Please familiarise yourself with the assessment due dates. If you need an extension, please contact your lecturer before the due date. If you require more than a week's extension, a medical certificate or support letter from your manager may be needed.

\section*{Resubmissions}
Students may be requested to resubmit an assessment following a rework of part/s of the original assessment. Resubmissions are completed within a short time frame (usually no more than 5 working days) and usually must be completed within the timing of the course to which the assessment relates. Resubmissions will be available to students who have made a genuine attempt at the first assessment opportunity. The maximum grade awarded for resubmission will be C-.

\section*{Learning Outcomes}
At the successful completion of this course, students will be able to:
\begin{enumerate}
	\item Demonstrate sound programming by following design patterns and best practices.
	\item Design and implement full-stack applications using industry relevant programming languages.
\end{enumerate}

\newpage

\section*{Assessment Overview - Learning Outcomes 1}
In this assessment, you will complete a series of programming tasks covering the lecture \& resource material. \\

\renewcommand{\arraystretch}{1.5}
\begin{tabular}{|c|c|c|c|} 
	\hline
	\textbf{Topic}                                              & \textbf{Weighting}  & \textbf{Due Date} \\ \hline 
	\small Python 1: Abstract Data Types \& OOP Recap           & \small 0.5\%        & \small 03-08-2020 at 5pm \\ \hline
	\small Python 2: More Abstract Data Types                   & \small 0.5\%        & \small 03-08-2020 at 5pm \\ \hline
	\small Python 3: Functional Programming                     & \small 1\%          & \small 10-08-2020 at 5pm \\ \hline
	\small Python 4: In-Built Functions \& SOLID                & \small 1\%          & \small 10-08-2020 at 5pm \\ \hline
	\small Python 5: Exceptions \& Automation Testing           & \small 1\%          & \small 17-08-2020 at 5pm \\ \hline
	\small Python 6: Concurrency \& Parallelism                 & \small 1\%          & \small 17-08-2020 at 5pm \\ \hline
	\small Django 1: Route, Model \& Admin Site                 & \small No Weighting &  \\ \hline
	\small Django 2: View \& Template                           & \small 1\%          & \small 24-08-2020 at 5pm \\ \hline
	\small Django 3: Forms \& Class-Based Views                 & \small 1\%          & \small 31-08-2020 at 5pm \\ \hline
	\small Django 4: Template Inheritance, Static Files \& CDNs & \small 1\%          & \small 31-08-2020 at 5pm \\ \hline
	\small Django 5: Automation Testing                         & \small 1\%          & \small 07-09-2020 at 5pm \\ \hline
	\small Django 6: Authentication                             & \small 1\%          & \small 07-09-2020 at 5pm \\ \hline
	\small Django 7: Security                                   & \small No Weighting & \\ \hline
	\small Django 8: Django REST Framework           & \small 1\%          & \small 14-09-2020 at 5pm \\ \hline
	\small Django 9: Deployment                                 & \small 1\%          & \small 21-09-2020 at 5pm \\ \hline
	\small React 1: Create-React-App  \& JSX                          & \small No Weighting &  \\ \hline
	\small React 2:  Components \& Props                                        & \small 1\%          & \small 28-09-2020 at 5pm \\ \hline
	\small React 3: State \& Lifecycle Methods                                       & \small 1\%          & \small 28-09-2020 at 5pm \\ \hline
	\small React 4: Events \& Conditional Rendering                      & \small 2\%          & \small 12-10-2020 at 5pm \\ \hline
	\small React 5:   Lists \& Keys                                     & \small No Weighting         & \small  \\ \hline
	\small React 6: Forms                          & \small 1\%          & \small 27-10-2020 at 5pm \\ \hline
	\small React 7: Automation Testing                            & \small 1\%          & \small 27-10-2020 at 5pm \\ \hline
	\small React 8: Deployment                       & \small 2\%          & \small 02-11-2020 at 5pm \\ \hline
\end{tabular}
\end{document}