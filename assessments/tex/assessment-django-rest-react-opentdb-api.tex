% Author: Grayson Orr
% Course: IN608: Intermediate Application Development Concepts

\documentclass{article}
\author{}

\usepackage{graphicx}
\usepackage{wrapfig}
\usepackage{enumerate}
\usepackage{hyperref}
\usepackage[margin = 2.25cm]{geometry}
\usepackage[table]{xcolor}
\usepackage{fancyhdr}
\hypersetup{
  colorlinks = true,
  urlcolor = blue
}
\setlength\parindent{0pt}
\pagestyle{fancy}
\fancyhf{}
\rhead{College of Engineering, Construction and Living Sciences\\Bachelor of Information Technology}
\lfoot{Django REST Framework, React \& OpenTDB API \\Version 2, 2021}
\rfoot{\thepage}

\begin{document}

\begin{figure}
	\centering
	\includegraphics[width=50mm]{./img/logo.png}
\end{figure}

\title{College of Engineering, Construction and Living Sciences\\Bachelor of Information Technology\\IN608: Intermediate Application Development Concepts\\Level 6, Credits 15\\\textbf{Django REST Framework, React \& OpenTDB API}}
\date{}
\maketitle

\section*{Assessment Overview} 
For this assessment, you will design, develop \& deploy a quiz tournament API using Django REST Framework, React, OpenTDB API \& Heroku. The main purpose of this assessment is not just to build a full-stack application, rather to demonstrate an ability to decouple the back-end from the front-end by creating two separate applications which interact with each other. Marks will be allocated for functionality \& best practices such as application robustness, code elegance, documentation \& git usage. \\

With the nation-wide lockdown over, your local pub is now able to run their weekly quiz tournament onsite. The online quiz tournament application proved to be a huge success \& the pub owners ask if you want to create a public API which allows users to create their own quiz tournaments.

\section*{Assessment Table}
\renewcommand{\arraystretch}{1.5}
\begin{tabular}{|l|l|l|l|l|}
	\hline		
	\vtop{\hbox{\strut \textbf{Assessment}}\hbox{\strut \textbf{Activity}}} & \textbf{Weighting} & \vtop{\hbox{\strut \textbf{Learning}}\hbox{\strut \textbf{Outcomes}}} & \vtop{\hbox{\strut \textbf{Assessment}}\hbox{\strut \textbf{Grading Scheme}}} & \vtop{\hbox{\strut \textbf{Completion}}\hbox{\strut \textbf{Requirements}}} \\
					
	\hline
						
	\small Practicals                                                       & \small 25\%        & \small 1                                                              & \small CRA                                                                    & \small Cumulative                                                           \\ \hline
	\small Django \& OpenTDB API                                            & \small 40\%        & \small 1, 2                                                           & \small CRA                                                                    & \small Cumulative                                                           \\ \hline
	\small Django REST Framework, React \& OpenTDB API                      & \small 35\%        & \small 1, 2                                                           & \small CRA                                                                    & \small Cumulative                                                           \\ \hline   
\end{tabular}

\section*{Conditions of Assessment}
This assessment will need to be completed by Wednesday, 23 June 2021 at 5pm. There will be availability during the teaching sessions to discuss the requirements \& progress of this assessment. 

\section*{Pass Criteria}
This assessment is criterion-referenced with a cumulative pass mark of 50\%.

\section*{Submission Details}
You must submit your program files via \textbf{GitHub Classroom}. Here is the link to the repository you will be using for your submission – \href{https://classroom.github.com/a/ww3bvOnY}{https://classroom.github.com/a/ww3bvOnY}.

\section*{Authenticity}
All parts of your submitted assessment must be completely your work and any references must be cited appropriately.

\section*{Policy on Submissions, Extensions, Resubmissions \& Resits}
The school's process concerning \textbf{Submissions, Extensions, Resubmissions and Resits} complies with Otago Polytechnic policies. Students can view policies on the Otago Polytechnic website located at \href{https://www.op.ac.nz/about-us/governance-and-management/policies}{https://www.op.ac.nz/about-us/governance-and-management/policies}.

\section*{Extensions}
Please familiarise yourself with the assessment due dates. If you need an extension, please contact your lecturer before the due date. If you require more than a week's extension, a medical certificate or support letter from your manager may be needed.

\section*{Resubmissions}
Students may be requested to resubmit an assessment following a rework of part/s of the original assessment. Resubmissions are completed within a short time frame (usually no more than 5 working days) and usually must be completed within the timing of the course to which the assessment relates. Resubmissions will be available to students who have made a genuine attempt at the first assessment opportunity. The maximum grade awarded for resubmission will be C-.

\section*{Learning Outcomes}
At the successful completion of this course, students will be able to:
\begin{enumerate}
	\item Demonstrate sound programming by following design patterns and best practices.
	\item Design and implement full-stack applications using industry relevant programming languages.
\end{enumerate}

\newpage

\section*{Instructions} 
This is a project-based assessment. Within your project you will need to implement the following:

\subsection*{Functionality \& Robustness - Learning Outcomes 1, 2}
\begin{itemize}
	\item Dependencies are correctly managed using \textbf{Pipenv}/\textbf{Pipfile} \& \textbf{npm}/\textbf{package.json}.
	\item Deploy both applications as one to Heroku.
	\begin{itemize}
		\item \textbf{Resource:} \href{https://librenepal.com/article/django-and-create-react-app-together-on-heroku/}{Deploying a Django + React App to Heroku}
	\end{itemize}
	\item Django REST Framework application (back-end):
	\begin{itemize}
		\item Create model classes which store the following quiz tournament data: creator, name, category, difficulty, question, correct answer \& incorrect answers. 
		\item Dynamically fetch \textbf{all} categories from the following URL \href{https://opentdb.com/api\_category.php}{https://opentdb.com/api\_category.php} \& store as choices in the appropriate model class.
		\item For each model class:
		\begin{itemize}
			\item Create a serializer class. 
			\item Create an \textbf{APIView} class or \textbf{api\_view} function which reads, inserts, updates \& deletes model data. \textbf{Hint:} use the \textbf{GET}, \textbf{POST}, \textbf{PUT} \& \textbf{DELETE} HTTP methods.
			\begin{itemize}
				\item \textbf{Resource:} \href{https://www.django-rest-framework.org/api-guide/views}{Django REST Framework Views}
			\end{itemize}
		\end{itemize} 
		\item Quiz tournament data is persistently stored in \textbf{Heroku PostgreSQL}.
		\begin{itemize}
			\item \textbf{Resource:} \href{https://www.heroku.com/postgres}{Heroku PostgreSQL}
		\end{itemize}
		\item Unit tests cover models, views \& OpenTDB API.
	\end{itemize}
	\item React application (front-end):
	\begin{itemize}
		\item Request quiz tournament data via \textbf{Django REST Framework} end-points using \textbf{Axios}. Data includes creator, name, category, difficulty, question, correct answer \& incorrect answers. 
		\item Create a new quiz tournament. Display a form in a modal. Form fields include creator, name, category \& difficulty. You \textbf{must} use the \textbf{select} input type for categories \& difficulties.  
		\item Incorrect formatted form field values handled gracefully using validation error messages, for example, creator form field is blank.
		\item View quiz tournaments in a table. Table data \textbf{must} includes creator, name, category, difficulty. Paginate quiz tournament data across several pages with \textbf{Next/Previous} links.
		\item Update a quiz tournament. Display a form in a modal. Form fields include creator, name, category \& difficulty.
		\item Delete a quiz tournament. Prompt the user for deletion.
		\item View a quiz tournament's question, correct answer \& incorrect answers when a quiz tournament name is clicked. 
		\item Visually attractive user-interface with a coherent graphical theme \& style using \textbf{Reactstrap}.
		\begin{itemize}
			\item \textbf{Resource:} \href{https://reactstrap.github.io/}{Reactstrap}
		\end{itemize}
		\item End-to-end tests cover creating, updating \& deleting a quiz tournament \& viewing a quiz tournament’s questions.
		\begin{itemize}
			\item \textbf{Resource:} \href{https://docs.cypress.io/guides/overview/why-cypress.html#In-a-nutshell}{Cypress.IO}
		\end{itemize}
	\end{itemize}
\end{itemize}

\subsection*{Documentation \& Git Usage - Learning Outcome 1}
\begin{itemize}
    \item Provide the following information in the repository \textbf{README.md} file:
    \begin{itemize}
		\item How do you set up the environment for development, i.e., after the repository is cloned, what do I need to start coding?
		\item How to run tests.
		\item How to deploy the applications.
		\item Link to the application on Heroku
    \end{itemize}
    \item At least 10 feature branches excluding the \textbf{main} branch.
    \begin{itemize}
        \item Your branches must be prefix with \textbf{feature}, for example, \textbf{feature-$<$name of functional requirement$>$}.
        \item For each branch, merge your own pull request to the \textbf{main} branch.
    \end{itemize}
    \item Commit messages must reflect the context of each functional requirement change.
    \begin{itemize}
		\item \textbf{Resource:} \href{https://www.freecodecamp.org/news/writing-good-commit-messages-a-practical-guide}{Writing Good Commit Messages}
	\end{itemize}
\end{itemize}

\end{document}