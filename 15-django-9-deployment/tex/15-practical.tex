\documentclass{article}
\author{}

\usepackage{graphicx}
\usepackage{wrapfig}
\usepackage{enumerate}
\usepackage{hyperref}
\usepackage{float}
\usepackage[margin = 2.25cm]{geometry}
\usepackage[table]{xcolor}
\usepackage{fancyhdr}
\hypersetup{
  colorlinks = true,
  urlcolor = blue
}
\setlength\parindent{0pt}
\pagestyle{fancy}
\fancyhf{}
\rhead{College of Engineering, Construction and Living Sciences\\Bachelor of Information Technology}
\lfoot{Practical 15 Django 9: Deployment \\Version 2, 2021}
\rfoot{\thepage}

\begin{document}

\begin{figure}
	\centering
	\includegraphics[width=50mm]{./img/logo.png}
\end{figure}

\title{College of Engineering, Construction and Living Sciences\\Bachelor of Information Technology\\IN608: Intermediate Application Development Concepts\\Level 6, Credits 15\\\textbf{Practical 15 Django 9: Deployment}} 
\date{}
\maketitle

\textbf{Due Date:} 07-05-2021 at 5pm \\

In this practical, you will complete a series of tasks covering today's lecture. This practical is worth 2\% of the final mark for the IN608: Intermediate Application Development Concepts course. \\

Before you start, in your practicals repository, create a new branch called \textbf{15-practical}.

\section*{Task} 
Deploy your \texttt{dog} Django project from \texttt{Practical 10 Django 4: Template Inheritance, Static Files \& CDNs} to \textbf{Heroku}.

\subsection*{Resources} 
\begin{itemize}
  \item \href{https://devcenter.heroku.com/articles/deploying-python}{Deploying Python + Django Apps on Heroku}
\end{itemize}

\end{document}